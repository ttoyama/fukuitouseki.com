% Options for packages loaded elsewhere
% Options for packages loaded elsewhere
\PassOptionsToPackage{unicode}{hyperref}
\PassOptionsToPackage{hyphens}{url}
\PassOptionsToPackage{dvipsnames,svgnames,x11names}{xcolor}
%
\documentclass[
  japanese,
  letterpaper,
  DIV=11,
  numbers=noendperiod]{scrartcl}
\usepackage{xcolor}
\usepackage[top=20mm,left=20mm,heightrounded]{geometry}
\usepackage{amsmath,amssymb}
\setcounter{secnumdepth}{5}
\usepackage{iftex}
\ifPDFTeX
  \usepackage[T1]{fontenc}
  \usepackage[utf8]{inputenc}
  \usepackage{textcomp} % provide euro and other symbols
\else % if luatex or xetex
  \usepackage{unicode-math} % this also loads fontspec
  \defaultfontfeatures{Scale=MatchLowercase}
  \defaultfontfeatures[\rmfamily]{Ligatures=TeX,Scale=1}
\fi
\usepackage{lmodern}
\ifPDFTeX\else
  % xetex/luatex font selection
  \setmainfont[]{Hiragino Kaku Gothic ProN}
  \setsansfont[]{Hiragino Kaku Gothic ProN}
  \setmonofont[]{Menlo}
\fi
% Use upquote if available, for straight quotes in verbatim environments
\IfFileExists{upquote.sty}{\usepackage{upquote}}{}
\IfFileExists{microtype.sty}{% use microtype if available
  \usepackage[]{microtype}
  \UseMicrotypeSet[protrusion]{basicmath} % disable protrusion for tt fonts
}{}
\makeatletter
\@ifundefined{KOMAClassName}{% if non-KOMA class
  \IfFileExists{parskip.sty}{%
    \usepackage{parskip}
  }{% else
    \setlength{\parindent}{0pt}
    \setlength{\parskip}{6pt plus 2pt minus 1pt}}
}{% if KOMA class
  \KOMAoptions{parskip=half}}
\makeatother
% Make \paragraph and \subparagraph free-standing
\makeatletter
\ifx\paragraph\undefined\else
  \let\oldparagraph\paragraph
  \renewcommand{\paragraph}{
    \@ifstar
      \xxxParagraphStar
      \xxxParagraphNoStar
  }
  \newcommand{\xxxParagraphStar}[1]{\oldparagraph*{#1}\mbox{}}
  \newcommand{\xxxParagraphNoStar}[1]{\oldparagraph{#1}\mbox{}}
\fi
\ifx\subparagraph\undefined\else
  \let\oldsubparagraph\subparagraph
  \renewcommand{\subparagraph}{
    \@ifstar
      \xxxSubParagraphStar
      \xxxSubParagraphNoStar
  }
  \newcommand{\xxxSubParagraphStar}[1]{\oldsubparagraph*{#1}\mbox{}}
  \newcommand{\xxxSubParagraphNoStar}[1]{\oldsubparagraph{#1}\mbox{}}
\fi
\makeatother


\usepackage{longtable,booktabs,array}
\usepackage{calc} % for calculating minipage widths
% Correct order of tables after \paragraph or \subparagraph
\usepackage{etoolbox}
\makeatletter
\patchcmd\longtable{\par}{\if@noskipsec\mbox{}\fi\par}{}{}
\makeatother
% Allow footnotes in longtable head/foot
\IfFileExists{footnotehyper.sty}{\usepackage{footnotehyper}}{\usepackage{footnote}}
\makesavenoteenv{longtable}
\usepackage{graphicx}
\makeatletter
\newsavebox\pandoc@box
\newcommand*\pandocbounded[1]{% scales image to fit in text height/width
  \sbox\pandoc@box{#1}%
  \Gscale@div\@tempa{\textheight}{\dimexpr\ht\pandoc@box+\dp\pandoc@box\relax}%
  \Gscale@div\@tempb{\linewidth}{\wd\pandoc@box}%
  \ifdim\@tempb\p@<\@tempa\p@\let\@tempa\@tempb\fi% select the smaller of both
  \ifdim\@tempa\p@<\p@\scalebox{\@tempa}{\usebox\pandoc@box}%
  \else\usebox{\pandoc@box}%
  \fi%
}
% Set default figure placement to htbp
\def\fps@figure{htbp}
\makeatother



\ifLuaTeX
\usepackage[bidi=basic,provide=*]{babel}
\else
\usepackage[bidi=default,provide=*]{babel}
\fi
\ifPDFTeX
\else
\babelfont{rm}[]{Hiragino Kaku Gothic ProN}
\fi
% get rid of language-specific shorthands (see #6817):
\let\LanguageShortHands\languageshorthands
\def\languageshorthands#1{}


\setlength{\emergencystretch}{3em} % prevent overfull lines

\providecommand{\tightlist}{%
  \setlength{\itemsep}{0pt}\setlength{\parskip}{0pt}}



 


\usepackage{xeCJK}
\setCJKmainfont{Hiragino Kaku Gothic ProN}
\KOMAoption{captions}{tableheading}
\makeatletter
\@ifpackageloaded{caption}{}{\usepackage{caption}}
\AtBeginDocument{%
\ifdefined\contentsname
  \renewcommand*\contentsname{目次}
\else
  \newcommand\contentsname{目次}
\fi
\ifdefined\listfigurename
  \renewcommand*\listfigurename{図一覧}
\else
  \newcommand\listfigurename{図一覧}
\fi
\ifdefined\listtablename
  \renewcommand*\listtablename{表一覧}
\else
  \newcommand\listtablename{表一覧}
\fi
\ifdefined\figurename
  \renewcommand*\figurename{図}
\else
  \newcommand\figurename{図}
\fi
\ifdefined\tablename
  \renewcommand*\tablename{表}
\else
  \newcommand\tablename{表}
\fi
}
\@ifpackageloaded{float}{}{\usepackage{float}}
\floatstyle{ruled}
\@ifundefined{c@chapter}{\newfloat{codelisting}{h}{lop}}{\newfloat{codelisting}{h}{lop}[chapter]}
\floatname{codelisting}{コード}
\newcommand*\listoflistings{\listof{codelisting}{コード一覧}}
\makeatother
\makeatletter
\makeatother
\makeatletter
\@ifpackageloaded{caption}{}{\usepackage{caption}}
\@ifpackageloaded{subcaption}{}{\usepackage{subcaption}}
\makeatother
\usepackage{bookmark}
\IfFileExists{xurl.sty}{\usepackage{xurl}}{} % add URL line breaks if available
\urlstyle{same}
\hypersetup{
  pdftitle={各施設の役割と対応},
  pdflang={ja-JP},
  colorlinks=true,
  linkcolor={blue},
  filecolor={Maroon},
  citecolor={Blue},
  urlcolor={Blue},
  pdfcreator={LaTeX via pandoc}}


\title{各施設の役割と対応}
\author{}
\date{}
\begin{document}
\maketitle

\renewcommand*\contentsname{目次}
{
\hypersetup{linkcolor=}
\setcounter{tocdepth}{3}
\tableofcontents
}

\section{各施設の役割と対応}\label{ux5404ux65bdux8a2dux306eux5f79ux5272ux3068ux5bfeux5fdc}

\subsection{目次}\label{ux76eeux6b21}

\begin{itemize}
\tightlist
\item
  \hyperref[ux6982ux8981]{概要}
\item
  \hyperref[ux5e73ux6642ux306eux6e96ux5099]{平時の準備}

  \begin{itemize}
  \tightlist
  \item
    \hyperref[ux65bdux8a2dux8a2dux5099ux306eux5b9aux671fux70b9ux691c]{施設設備の定期点検}
  \item
    \hyperref[ux5099ux84c4ux7269ux8cc7ux306eux5b9aux671fux70b9ux691c]{備蓄物資の定期点検}
  \item
    \hyperref[ux707dux5bb3ux5bfeux5fdcux8a13ux7df4ux306eux5b9fux65bdux3068ux30deux30cbux30e5ux30a2ux30ebux7fd2ux719f]{災害対応訓練の実施とマニュアル習熟}
  \item
    \hyperref[ux60a3ux8005ux5bb6ux65cfux3078ux306eux4e8bux524dux8aacux660eux3068ux7dcaux6025ux9023ux7d61ux5148ux78baux8a8d]{患者・家族への事前説明と緊急連絡先確認}
  \item
    \hyperref[ux4ea4ux901aux30a2ux30afux30bbux30b9ux78baux4fddux306eux305fux3081ux306eux5e73ux5e38ux6642ux304bux3089ux306eux6e96ux5099]{交通アクセス確保のための平常時からの準備}
  \end{itemize}
\item
  \hyperref[ux767aux707dux76f4ux5f8cux306eux5bfeux5fdc]{発災直後の対応}

  \begin{itemize}
  \tightlist
  \item
    \hyperref[ux60a3ux8005ux5b89ux5168ux78baux8a8dux30b9ux30bfux30c3ux30d5ux5b89ux5426ux78baux8a8d]{患者安全確認、スタッフ安否確認}
  \item
    \hyperref[ux65bdux8a2dux8a2dux5099ux70b9ux691c]{施設設備点検}
  \item
    \hyperref[ux4ea4ux901aux30a2ux30afux30bbux30b9ux72b6ux6cc1ux78baux8a8d]{交通アクセス状況確認}
  \item
    \hyperref[ux707dux5bb3ux6642ux30a2ux30afux30bbux30b9ux78baux4fddux306bux95a2ux3059ux308bux57faux672cux65b9ux91ddux3068ux884cux52d5ux8a08ux753b]{災害時アクセス確保に関する基本方針と行動計画}
  \end{itemize}
\item
  \hyperref[ux767aux707d2ux9031ux9593ux306eux5bfeux5fdc]{発災〜2週間の対応}

  \begin{itemize}
  \tightlist
  \item
    \hyperref[ux900fux6790ux5b9fux65bdux53efux80fdux65bdux8a2dux306eux5bfeux5fdc]{透析実施可能施設の対応}
  \item
    \hyperref[ux900fux6790ux5b9fux65bdux56f0ux96e3ux65bdux8a2dux306eux5bfeux5fdc]{透析実施困難施設の対応}
  \item
    \hyperref[ux5171ux901aux5bfeux5fdcux4e8bux9805]{共通対応事項}
  \end{itemize}
\item
  \hyperref[2ux9031ux4ee5ux964dux306eux5bfeux5fdc]{2週以降の対応}

  \begin{itemize}
  \tightlist
  \item
    \hyperref[ux65bdux8a2dux306eux672cux683cux5fa9ux65e7]{施設の本格復旧}
  \item
    \hyperref[ux7d99ux7d9aux7684ux652fux63f4ux4f53ux5236]{継続的支援体制}
  \end{itemize}
\end{itemize}

\subsection{概要}\label{ux6982ux8981}

災害発生時から復旧に至るまでの各透析施設における役割と対応について、時系列に沿って記載します。各施設は被災状況と稼働能力に応じて、適切な対応を実施してください。

\subsubsection{透析患者の総合的な災害対策}\label{ux900fux6790ux60a3ux8005ux306eux7dcfux5408ux7684ux306aux707dux5bb3ux5bfeux7b56}

透析患者とその家族は、透析治療特有の対策に加えて、一般的な災害対策も併せて実施することが重要です。

\textbf{国の基本的な防災対策}\\
政府広報オンライン「【防災特集】災害への備えを、日本の標準装備に。」では、全国民が備えるべき標準的な災害対策が示されています。透析患者も、まずはこれらの基本的な防災準備を確実に実施しましょう。\\
URL: https://www.gov-online.go.jp/tokusyu/bousai/

\textbf{福井県の防災情報}\\
福井県では、地域の特性に応じた防災情報やリアルタイムの災害情報を提供しています。

\begin{itemize}
\item
  福井県危機対策・防災情報ポータルサイト\\
  URL: https://www.pref.fukui.lg.jp/doc/kikitaisaku/portalsite.html
\item
  福井県防災ネット(防災情報総合サイト)\\
  URL: https://www.bousai.pref.fukui.lg.jp/dis\_portal/index.html
\end{itemize}

\subsubsection{防災情報サイト
QRコード}\label{ux9632ux707dux60c5ux5831ux30b5ux30a4ux30c8-qrux30b3ux30fcux30c9}

災害現場で素早く防災情報にアクセスできるよう、各サイトのQRコードを掲載します。

政府広報防災特集 国の基本防災対策

福井県危機対策ポータル 県の防災情報

福井県防災ネット リアルタイム防災情報

\subsection{平時の準備}\label{ux5e73ux6642ux306eux6e96ux5099}

各透析施設が災害発生に備えて平常時から実施すべき準備について記載します。

\subsubsection{施設設備の定期点検}\label{ux65bdux8a2dux8a2dux5099ux306eux5b9aux671fux70b9ux691c}

→
\textbf{チェックシート111参照}:\href{1311_施設設備点検表.qmd}{施設設備点検表}

\textbf{施設及び設備の定期的な自己点検}

各施設で作成されたマニュアルに基づき、施設及び設備の定期的な自己点検を実施します。

火災警報やスプリンクラー、エレベーターなど、一般的な災害に備えた防災機能についても定期的に点検します。

\textbf{ライフラインの点検と確保}

\begin{itemize}
\tightlist
\item
  医療機関の維持に必要な透析機器、電気、水道、燃料などの施設・設備の点検を平常時から定期的に実施します。
\item
  水道事務所や電力会社等の担当部門、またはビルの所有者等と相談し、透析用の水、電力等の確保の方法を確認しておきます。
\item
  災害発生時に備え、電気、水、燃料、食料、医薬品、医療用器材などの調達方法について、電力会社、水道事務所、ガス会社、取引先業者等と、あらかじめ調整や確認を行っておきます。
\item
  可能であれば、水の確保のために貯水槽、貯水タンク、自家発電装置を準備します。
\end{itemize}

\textbf{透析装置等の転倒防止対策}

\begin{itemize}
\tightlist
\item
  透析液作成装置は、転倒、移動、揺れによる損傷を防止するために、床面にアンカーボルト等でしっかり固定します。
\item
  施設が耐震構造の場合は透析用監視装置(コンソール)のキャスターはロックしないでフリーにし、透析ベッドのキャスターはロックし、コンソールとベッドが離れないようにバックルベルト等で連結することでラインの抜去を免れるようにすると良いとされています。
\item
  施設が免振構造の場合は透析用監視装置(コンソール)のキャスターはロックし、透析ベッドのキャスターもロックすると良いとされています。
\item
  透析用給水に用いられる塩化ビニル管は破損しやすいため、接続部分をフレキシブル管へ変更するなどの対策をとります。また、破損しても修復しやすい材料を選定することも考慮されます。
\end{itemize}

\textbf{緊急対応物品の整備と設置}

停電時用懐中電灯、情報収集用携帯テレビまたはラジオ、患者誘導用ハンドマイクなどの用品をすぐ取り出せる場所に収納し、スタッフに周知します。

\subsubsection{備蓄物資の定期点検}\label{ux5099ux84c4ux7269ux8cc7ux306eux5b9aux671fux70b9ux691c}

→
\textbf{チェックシート112参照}:\href{1320_備蓄物資点検チェックシート.qmd}{備蓄物資点検チェックシート}

\textbf{医療機関が行う備蓄と点検}

\begin{itemize}
\tightlist
\item
  平常時から、医療機関の維持に必要な透析機器や、電気、水道などの施設・設備の点検を定期的に実施し、耐震性の確保や患者の安全確保に努めます。
\item
  ダイアライザー・回路等の透析器材、透析液、透析に必要な医薬品について、可能な限りの備蓄に努めます。
\item
  災害発生時の電気、水、燃料、食料、医薬品、医療用器材などの調達方法について、電力会社、水道事務所、ガス会社、取引先業者等と、あらかじめ調整や確認を行っておきます。
\end{itemize}

\subsubsection{災害対応訓練の実施とマニュアル習熟}\label{ux707dux5bb3ux5bfeux5fdcux8a13ux7df4ux306eux5b9fux65bdux3068ux30deux30cbux30e5ux30a2ux30ebux7fd2ux719f}

\textbf{災害対策マニュアルの作成と内容}

\begin{itemize}
\tightlist
\item
  医療機関は、院内に災害対策委員会を設置し、災害時の対応をまとめたマニュアルを作成しておく必要があります。
\item
  各透析医療機関の実態に即して作成し、日頃から訓練や確認を行い、災害時に混乱しないようにします。
\end{itemize}

マニュアルには、以下の事項を具体的に分かりやすく記載します:

\begin{itemize}
\tightlist
\item
  指揮系統の確立(管理者不在時の代理者の設定など)
\item
  患者、行政機関、透析医療機関間の情報収集と指示伝達の手段の確立
\item
  情報と指示の流れの確認
\item
  緊急離脱の判断、方法の取り決め
\item
  患者移送手段の確保
\item
  防災の観点による避難経路、建物・透析設備の見直し
\item
  災害時の水道・電気・ガス・医療資材などの確保
\item
  持ち出すべき物品、救急処置物品、救急カート、AED、血圧計など
\item
  防災訓練の実施
\end{itemize}

「透析患者向けマニュアル」は別途作成し、連絡手段、非常口や避難経路、避難方法に関する情報を提供しておくことが推奨されます。

\textbf{災害対応訓練の実施}

\begin{itemize}
\tightlist
\item
  災害対策委員会は定期的に開催され、防災意識の共有、患者及び職員の教育、防災訓練などを行うことが求められます。
\item
  作成したマニュアルに基づき、防災訓練の実施と施設及び設備の定期的な自己点検を行います。
\item
  定期的かつ計画的な防災訓練の実施は、大規模災害発生時に安全に避難し、円滑な医療救護活動を展開するために不可欠です。
\end{itemize}

訓練の内容には、以下が含まれるべきです:

\begin{itemize}
\tightlist
\item
  \textbf{避難訓練}:事前に避難訓練を行い、ストレッチャーや車椅子などの介護を要する患者にも対応した避難の知識や介護の技術の習得
\item
  \textbf{情報収集・伝達訓練}:通信機器を用いた、より実践的な災害時情報収集〜情報伝達訓練を実施し、通信網が使用不可能となる可能性も考慮する
\item
  \textbf{トリアージ体制の訓練}:大災害を想定したトリアージ体制の訓練
\item
  \textbf{緊急離脱訓練}:各施設のマニュアル従った緊急離脱の判断と、判断に従った離脱訓練
\end{itemize}

\textbf{職員の習熟}

\begin{itemize}
\tightlist
\item
  日頃から安全確保に留意した透析技術の向上に努め、職員全員が設備や機器などの取り扱いに習熟します。
\item
  停電時などに患者監視装置が停止した場合に、体外に出ている血液が凝固する前に迅速にバッテリー電源への切り替えを行うため、平常時より落差回収法やポンプ手動回収法などに慣れておくことも大切です。
\end{itemize}

\subsubsection{患者・家族への事前説明と緊急連絡先確認}\label{ux60a3ux8005ux5bb6ux65cfux3078ux306eux4e8bux524dux8aacux660eux3068ux7dcaux6025ux9023ux7d61ux5148ux78baux8a8d}

→
\textbf{チェックシート114参照}:\href{1330_患者緊急連絡先チェックシート.qmd}{患者緊急連絡先チェックシート}

\textbf{自己管理の徹底}

\begin{itemize}
\tightlist
\item
  災害時には透析不足となることが予測されるため、日頃から体重や食事管理、薬の内服など自己管理を適切に行えるよう患者を指導します。
\item
  特に、災害時に透析間隔が開いてしまう場合の生活上の注意点や、通常の治療食がとれない場合に備えて、避難所での配給食のうち何を食べても良いか、食べてはいけないのか実践的な指導を行っておきます。
\item
  避難所などで支給されそうな食品では塩分やカリウム含有量の多い食品に注意しましょう。詳細な食品栄養成分については、\href{9930_透析患者向け食品栄養成分表.qmd}{透析患者向け食品栄養成分表}を参照してください。
\end{itemize}

\textbf{緊急時の行動と避難方法}

\begin{itemize}
\tightlist
\item
  災害時、透析中の患者には、穿刺針が抜けないように血液回路をしっかり握り、ベッドの柵につかまって振り落とされないように指導します。
\item
  毛布をかぶって蛍光灯などの落下物を防ぐよう伝えます。
\item
  透析中止及び避難の指示が出た場合の「血液回路からの離脱方法」に従って離脱するよう指導します。
\item
  スタッフの誘導に従って施設指定の避難場所に避難し、安否を報告するよう伝えます。勝手な行動はスタッフが安否を気遣い探すことがあることを伝えておきます。
\item
  患者には非常口や避難経路、避難方法、避難場所に関して情報を提供しておきます。
\end{itemize}

\textbf{緊急連絡先の確認}

透析医療機関は、透析が実施可能か否かを知らせるために、患者・家族の緊急連絡先を把握しておくことが大切です。

\subsubsection{交通アクセス確保のための平常時からの準備}\label{ux4ea4ux901aux30a2ux30afux30bbux30b9ux78baux4fddux306eux305fux3081ux306eux5e73ux5e38ux6642ux304bux3089ux306eux6e96ux5099}

\textbf{主要アクセス経路の把握と代替経路の検討}

\begin{itemize}
\tightlist
\item
  各施設は、自施設への主要アクセス経路を複数確認し、災害時の代替経路を事前に地図上で検討しておく。
\item
  公共交通機関の寸断も想定し、自家用車でのアクセス経路も確認する。
\end{itemize}

\textbf{職員の参集経路の確認}

\begin{itemize}
\tightlist
\item
  職員の自宅から施設までの複数の参集経路を事前に確認し、共有しておく。
\item
  地域によっては、徒歩や自転車での参集も想定する。
\end{itemize}

\textbf{物資輸送体制の確認}

透析液や医薬品などの納入業者との間で、災害時の輸送体制(代替ルート、緊急配送、備蓄等)について事前に協議しておく。

\subsection{発災直後の対応}\label{ux767aux707dux76f4ux5f8cux306eux5bfeux5fdc}

災害発生後早期に全施設共通で実施すべき初動対応について記載します。

→ \textbf{初動対応チェックシート参照}: -
\href{1410_初動対応チェックシート_0-30分.qmd}{初動対応チェックシート\_0-30分}
-
\href{1420_初動対応チェックシート_1-3時間.qmd}{初動対応チェックシート\_1-3時間}
-
\href{1430_初動対応チェックシート_3-24時間.qmd}{初動対応チェックシート\_3-24時間}

患者が災害発生時に取るべき対応については、各施設の事情に合わせた施設ごとのマニュアルを整備してください。

\subsubsection{患者安全確認、スタッフ安否確認}\label{ux60a3ux8005ux5b89ux5168ux78baux8a8dux30b9ux30bfux30c3ux30d5ux5b89ux5426ux78baux8a8d}

→
\textbf{チェックシート121参照}:\href{1452_患者安全・スタッフ安否チェックシート.qmd}{患者安全・スタッフ安否チェックシート}

\textbf{患者安全確認}

\begin{itemize}
\tightlist
\item
  災害発生時、施設内にいる透析中の患者様や待機中の患者様の安全を最優先に確保します。
\item
  負傷者の有無を速やかに確認し、応急処置を行います。
\item
  強い揺れが収まったら、パニックを起こしそうになっている患者のそばに駆け寄り、安心感を与え、落ち着かせるなど心理的ケアを行います。
\item
  透析治療中の場合は、安全確保を最優先に透析の中止または継続を判断します。
\end{itemize}

\textbf{スタッフ安否確認}

\begin{itemize}
\tightlist
\item
  事前に定めた緊急連絡網や安否確認システム(例:一斉メール、SNSグループ、電話連絡網)を活用し、全スタッフの安否を速やかに確認します。
\item
  参集可能なスタッフには、安全に留意しつつ速やかな施設への参集を要請します。
\end{itemize}

\textbf{ネットワーク本部への報告}

\begin{itemize}
\tightlist
\item
  \textbf{安否確認の実施}:透析患者、スタッフの安否を迅速に確認する
\item
  \textbf{施設状況の点検}:建物、設備、ライフラインの被害状況を調査し、透析実施可否を判断する
\item
  \textbf{ネットワーク本部への報告}:施設状況、患者情報、支援要請を速やかに報告する
\item
  \textbf{患者・家族への情報提供}:透析実施状況、代替手段について情報を提供する
\item
  \textbf{緊急透析の実施}:可能な限り緊急透析を実施し、実施困難な場合は他施設搬送を検討する(→02章参照)
\end{itemize}

\subsubsection{施設設備点検}\label{ux65bdux8a2dux8a2dux5099ux70b9ux691c}

→
\textbf{チェックシート122参照}:\href{1461_施設設備点検チェックシート_発災直後.qmd}{施設設備点検チェックシート\_発災直後}

\textbf{建物・設備点検}

\begin{itemize}
\tightlist
\item
  施設の建物本体、透析装置、水処理装置、自家発電設備、医療ガス設備、医薬品・医療材料の保管状況など、透析医療に必要な主要な設備に損傷がないか、速やかに点検します。
\item
  特に、電力、水道、ガス、通信といったライフラインの状況を詳細に確認します。
\end{itemize}

\textbf{透析関連機器の確認}

\begin{itemize}
\tightlist
\item
  透析装置の稼働状況、警報発生の有無を確認し、安全機能の動作状況を点検します。
\item
  水処理装置の異常、透析液供給システムの損傷がないか確認します。
\end{itemize}

\subsubsection{交通アクセス状況確認}\label{ux4ea4ux901aux30a2ux30afux30bbux30b9ux72b6ux6cc1ux78baux8a8d}

→
\textbf{チェックシート123参照}:\href{1470_交通アクセス状況チェックシート.qmd}{交通アクセス状況チェックシート}

\begin{itemize}
\tightlist
\item
  施設周辺の道路状況(がけ崩れ、冠水、橋梁の損壊など)や、公共交通機関の運行状況を確認し、スタッフの参集や物資の搬入、患者の来院・搬送の可否を判断します。
\item
  福井県透析施設災害時アクセス確保に関する基本方針と行動計画に基づき、緊急車両の通行ルートや指定された優先道路の状況を把握します。
\end{itemize}

\subsubsection{災害時アクセス確保に関する基本方針と行動計画}\label{ux707dux5bb3ux6642ux30a2ux30afux30bbux30b9ux78baux4fddux306bux95a2ux3059ux308bux57faux672cux65b9ux91ddux3068ux884cux52d5ux8a08ux753b}

\textbf{基本方針}

\begin{itemize}
\tightlist
\item
  \textbf{患者の安全確保と治療継続}:災害時においても、患者さんの安全を最優先し、透析治療が途切れないよう努めます。
\item
  \textbf{情報収集と共有}:正確な情報を迅速に集め、関係者間で共有を徹底します。
\item
  \textbf{関係機関との連携}:福井県庁、市町村、自衛隊、警察、消防、医療機関など、あらゆる関係機関と密に連携します。
\item
  \textbf{アクセス経路の確保}:事前にアクセス経路を確認し、代替経路も検討しておきます。
\end{itemize}

\textbf{患者カテゴリー別優先度設定}

限られた医療資源の中で、透析治療の優先順位を設定します。

\begin{itemize}
\tightlist
\item
  \textbf{緊急性の高い患者}:高カリウム血症、肺水腫、呼吸困難、意識レベルの変化など生命に直結する合併症のある患者。
\item
  \textbf{最終透析からの経過時間}:最終透析からの時間が長い患者(例:96時間以上経過している患者)。
\item
  \textbf{残存腎機能の有無}:残存腎機能がほとんどない患者。
\item
  \textbf{併存疾患}:基礎疾患や合併症の重症度。
\item
  \textbf{透析治療法}:在宅での自己管理が可能な腹膜透析(PD)患者は優先順位が変動する可能性があります。
\item
  \textbf{トリアージの実施}:限られた医療資源の中で、トリアージの考え方に基づき、最も緊急性の高い患者(例:重症度が高い「赤」、中等症の「黄」に該当する患者)から優先的に透析を実施します。
\end{itemize}

\subsection{発災〜2週間の対応}\label{ux767aux707d2ux9031ux9593ux306eux5bfeux5fdc}

発災から2週間以内の応急期における、各施設の状況に応じた具体的な対応について記載します。この期間は緊急対応から復旧に向けた体制整備への転換期となります。

\subsubsection{透析実施可能施設の対応}\label{ux900fux6790ux5b9fux65bdux53efux80fdux65bdux8a2dux306eux5bfeux5fdc}

\textbf{受け入れ体制整備}

\begin{itemize}
\tightlist
\item
  \textbf{患者情報確認}:受け入れ患者の透析条件、シャント肢、既往歴、内服薬、緊急連絡先などを事前に確認し、受け入れ準備を整えます。
\item
  \textbf{ベッド・機器の確保}:可能な限り多くの透析ベッドと機器を稼働させ、受け入れ能力を最大化します。
\item
  \textbf{人員配置}:応援職員の受け入れや、自施設職員の柔軟なシフト調整により、必要な人員を確保します。
\item
  \textbf{スペースの確保}:患者待機場所、医療資材保管場所、職員休憩場所など、必要なスペースを確保します。
\item
  \textbf{情報提供}:受け入れ可能人数、透析可能時間、アクセス方法などを、県災害対策本部、福井県透析医会、近隣施設に速やかに情報提供します。
\end{itemize}

\textbf{感染対策}

\begin{itemize}
\tightlist
\item
  \textbf{標準予防策の徹底}:災害時においても、手洗い、サージカルマスク、手袋、エプロンなどの個人防護具の適切な使用を徹底します。
\item
  \textbf{環境整備}:透析室内の清掃・消毒を徹底し、感染リスクを最小限に抑えます。特に、水害発生時は浸水した場所の消毒を徹底します。
\item
  \textbf{感染症患者への対応}:発熱や呼吸器症状のある患者、感染症が疑われる患者については、ゾーニングや隔離、専用の機器使用など、適切な感染対策を講じます。
\item
  \textbf{医療廃棄物処理}:感染性廃棄物と非感染性廃棄物の分別を徹底し、適切に処理します。処理が困難な場合は、一時的な安全な保管場所を確保し、行政と連携して回収方法を調整します。
\end{itemize}

\subsubsection{透析実施困難施設の対応}\label{ux900fux6790ux5b9fux65bdux56f0ux96e3ux65bdux8a2dux306eux5bfeux5fdc}

→
\textbf{チェックシート132参照}:\href{1431_透析実施困難施設対応チェックシート.qmd}{透析実施困難施設対応チェックシート}

\textbf{患者搬送調整}

\begin{itemize}
\tightlist
\item
  \textbf{患者安否確認と病態把握}:全透析患者の安否を確認し、透析緊急度(最終透析日、体液量、電解質異常など)を把握します。
\item
  \textbf{搬送先施設の選定}:福井県透析施設ネットワークや県災害対策本部からの情報に基づき、受け入れ可能な透析実施施設を選定します。
\item
  \textbf{搬送手段の確保}:患者の病態に応じた搬送手段(自家用車、救急車、消防車、DMAT車両など)を行政や消防と連携して確保します。
\item
  \textbf{情報伝達}:搬送先の施設に対し、患者情報(透析条件、シャント肢、既往歴、内服薬、緊急連絡先など)を正確かつ速やかに伝達します。
\item
  \textbf{家族への説明}:患者および家族に対し、搬送の必要性、搬送先、今後の治療方針について丁寧に説明し、同意を得ます。
\end{itemize}

\subsubsection{共通対応事項}\label{ux5171ux901aux5bfeux5fdcux4e8bux9805}

→
\textbf{チェックシート133参照}:\href{1321_物資薬剤管理チェックシート.qmd}{物資薬剤管理チェックシート}

\textbf{物資・薬剤管理}

\begin{itemize}
\tightlist
\item
  \textbf{在庫管理の強化}:透析液、回路、ダイアライザー、医薬品などの消費量を正確に把握し、残量を適切に管理します。
\item
  \textbf{調達計画の策定}:通常の納入ルートが使用できない場合を想定し、代替の調達先や調達方法を検討します。
\item
  \textbf{備蓄物資の活用}:事前に備蓄していた物資を計画的に活用し、必要に応じて他施設との物資融通を行います。
\end{itemize}

\textbf{情報共有と記録}

\begin{itemize}
\tightlist
\item
  \textbf{継続的な情報提供}:施設の稼働状況、患者受け入れ状況、物資の不足状況などを、定期的に関係機関に報告します。
\item
  \textbf{対応記録の作成}:災害発生からの対応状況、判断内容、患者の状況変化などを詳細に記録し、今後の対応や検証に活用します。
\end{itemize}

\subsection{2週以降の対応}\label{ux9031ux4ee5ux964dux306eux5bfeux5fdc}

災害発生から2週間以降の中長期的な復旧・復興期における施設対応について記載します。

\subsubsection{施設の本格復旧}\label{ux65bdux8a2dux306eux672cux683cux5fa9ux65e7}

\textbf{設備復旧計画}

\begin{itemize}
\tightlist
\item
  \textbf{詳細損害評価}:専門業者と連携し、透析装置、水処理装置、建物設備の詳細な損害評価を実施します。
\item
  \textbf{復旧優先順位の決定}:医療継続に必要な設備から優先的に復旧を進めます。
\item
  \textbf{代替設備の検討}:復旧に長期間を要する設備については、代替設備の導入や他施設との連携を検討します。
\end{itemize}

\textbf{人員体制の正常化}

\begin{itemize}
\tightlist
\item
  \textbf{職員の健康管理}:長期間の災害対応で疲労したスタッフの健康管理と休養体制を確保します。
\item
  \textbf{応援職員との調整}:応援職員との役割分担を明確にし、効率的な運営体制を構築します。
\item
  \textbf{研修・教育の実施}:災害対応の経験を踏まえた研修や教育を実施し、今後の対応力向上を図ります。
\end{itemize}

\subsubsection{継続的支援体制}\label{ux7d99ux7d9aux7684ux652fux63f4ux4f53ux5236}

\textbf{患者ケアの継続}

\begin{itemize}
\tightlist
\item
  \textbf{心理的ケア}:災害による心理的影響を受けた患者へのカウンセリングやサポートを提供します。
\item
  \textbf{生活支援}:避難生活を続ける患者への生活支援や相談対応を実施します。
\item
  \textbf{透析条件の最適化}:災害時の制限的な透析から、患者の状態に応じた最適な透析条件への調整を行います。
\end{itemize}

\textbf{地域連携の強化}

\begin{itemize}
\tightlist
\item
  \textbf{他施設との連携継続}:災害時に構築した施設間連携を継続し、平常時の協力体制を強化します。
\item
  \textbf{行政との連携}:復旧・復興過程で必要な行政支援の継続的な要請と調整を行います。
\item
  \textbf{患者の帰還受け入れ}:他施設に避難していた患者の帰還を受け入れる準備をする(→02章と連携)
\item
  \textbf{スタッフのケア}:長期間の災害対応を行ったスタッフの心身のケアを実施する
\end{itemize}

\begin{center}\rule{0.5\linewidth}{0.5pt}\end{center}

\href{index.qmd}{目次} \textbar{}
\href{0200_福井県透析施設ネットワークの役割と対応.qmd}{次:ネットワークの役割と対応
→}




\end{document}
