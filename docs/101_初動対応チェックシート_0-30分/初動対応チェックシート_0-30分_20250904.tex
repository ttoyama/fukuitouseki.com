% 初動対応チェックシート(発災後0~30分)
% LaTeX template for disaster dialysis manual forms

\documentclass[a4paper,12pt]{jarticle}
\usepackage[top=30mm, bottom=30mm, left=20mm, right=20mm, footskip=18mm, headsep=12mm]{geometry}
\usepackage{setspace}
\setstretch{1.3}
\usepackage{array}
\usepackage{longtable}
\usepackage{amssymb}
\usepackage{multirow}
\usepackage{booktabs}

\newcommand{\checkbox}{$\square$\ }
\newcommand{\checkedbox}{$\blacksquare$\ }
\newcommand{\underlinespace}[1]{\underline{\hspace{#1}}}
\newcommand{\circlecheck}{$\bigcirc$\ }

% シンプルなページ番号設定
\pagestyle{plain}
\makeatletter
\def\@oddhead{\hfill\small 101 初動対応チェックシート(0-30分) 2025.09.04版}
\def\@evenhead{\hfill\small 101 初動対応チェックシート(0-30分) 2025.09.04版}
\def\@oddfoot{\hfil\thepage\hfil}
\def\@evenfoot{\hfil\thepage\hfil}
\makeatother

\begin{document}

% 手動でタイトルを作成
\begin{center}
{\Large\textbf{初動対応チェックシート(発災後0~30分)}}
\end{center}
\vspace{5mm}

\noindent
\textbf{施設名:} \underlinespace{8cm}

\vspace{3mm}

\noindent
\textbf{記録者:} \underlinespace{4cm}

\vspace{3mm}

\noindent
\textbf{発災日時:} \underlinespace{2cm}年\underlinespace{1cm}月\underlinespace{1cm}日 \quad \circlecheck 午前 \quad \circlecheck 午後 \quad \underlinespace{1cm}時\underlinespace{1cm}分

\vspace{5mm}

\section*{1. 患者・スタッフ安全確保}

\subsection*{1-1. 施設内患者安全確保}

\checkbox \textbf{施設内患者(透析中・待機中)の安全を確保する}

\quad 確認時刻:\underlinespace{1cm}時\underlinespace{1cm}分 \quad 対応者:\underlinespace{4cm}

\vspace{3mm}

\checkbox \textbf{負傷者の有無を確認し、応急処置を実施する}

\quad 負傷者数:\underlinespace{2cm}人 \quad 処置内容:\underlinespace{6cm}

\vspace{3mm}

\checkbox \textbf{パニック患者のそばに駆け寄り、心理的ケアを行う}

\quad 対象患者数:\underlinespace{2cm}人 \quad 対応者:\underlinespace{4cm}

\vspace{3mm}

\checkbox \textbf{透析中患者の治療中止・継続を判断する}

\quad 透析中患者数:\underlinespace{2cm}人 \quad 継続:\underlinespace{2cm}人 \quad 中止:\underlinespace{2cm}人

\vspace{5mm}

\subsection*{1-2. スタッフ安否確認}

\checkbox \textbf{緊急連絡網を使用して全スタッフの安否を確認する}

\quad 確認完了時刻:\underlinespace{1cm}時\underlinespace{1cm}分

\vspace{3mm}

\checkbox \textbf{参集可能スタッフに速やかな施設参集を要請する}

\quad 要請完了時刻:\underlinespace{1cm}時\underlinespace{1cm}分

\vspace{5mm}

\section*{2. 施設設備点検}

\checkbox \textbf{建物本体の損傷を点検する}

\quad \circlecheck 安全 \quad \circlecheck 要注意 \quad \circlecheck 危険 \quad 点検者:\underlinespace{4cm}

\vspace{3mm}

\checkbox \textbf{透析装置の稼働状況・警報発生の有無を確認する}

\quad 稼働可能台数:\underlinespace{2cm}台/総\underlinespace{2cm}台 \quad 警報発生:\checkbox あり \checkbox なし

\vspace{3mm}

\checkbox \textbf{水処理装置の異常を確認する}

\quad \circlecheck 正常 \quad \circlecheck 異常あり(詳細:\underlinespace{6cm})

\vspace{3mm}

\checkbox \textbf{自家発電設備の動作状況を点検する}

\quad \circlecheck 正常 \quad \circlecheck 異常あり \quad \circlecheck 未確認 \quad 燃料残量:\underlinespace{3cm}

\vspace{3mm}

\checkbox \textbf{医療ガス設備を確認する}

\quad \circlecheck 正常 \quad \circlecheck 異常あり(詳細:\underlinespace{6cm})

\vspace{3mm}

\checkbox \textbf{医薬品・医療材料の保管状況を確認する}

\quad \circlecheck 問題なし \quad \circlecheck 破損あり(詳細:\underlinespace{6cm})

\vspace{3mm}

\checkbox \textbf{ライフライン状況を詳細に確認する}

\quad \textbf{電力:} \circlecheck 正常 \circlecheck 停電 \quad \textbf{水道:} \circlecheck 正常 \circlecheck 断水

\quad \textbf{ガス:} \circlecheck 正常 \circlecheck 停止 \quad \textbf{通信:} \circlecheck 正常 \circlecheck 不通

\vspace{5mm}

\newpage

\section*{3. 初動報告(発災後30分以内)}

\subsection*{3-1. 福井県透析施設ネットワーク本部への報告}

\checkbox \textbf{施設職員・患者の安否確認結果を報告する}

\quad 報告完了時刻:\underlinespace{1cm}時\underlinespace{1cm}分 \quad 報告者:\underlinespace{4cm}

\quad 患者安否:安全\underlinespace{2cm}人、負傷\underlinespace{2cm}人、不明\underlinespace{2cm}人

\quad スタッフ安否:安全\underlinespace{2cm}人、負傷\underlinespace{2cm}人、不明\underlinespace{2cm}人

\vspace{4mm}

\checkbox \textbf{建物の緊急危険度を判定して報告する}

\quad \circlecheck 安全(継続使用可能) \quad \circlecheck 要注意(一部制限) \quad \circlecheck 危険(使用不可)

\vspace{4mm}

\checkbox \textbf{即座に必要な緊急支援の有無を報告する}

\quad \checkbox 緊急支援不要 \quad \checkbox 緊急支援必要

\quad 必要支援内容:\underlinespace{10cm}

\vspace{5mm}

\subsection*{3-2. 詳細状況確認(30分~1時間の準備)}

\checkbox \textbf{施設周辺道路状況を確認する}

\quad 主要アクセス路:\circlecheck 通行可能 \circlecheck 通行困難 \circlecheck 通行不可

\quad 詳細:\underlinespace{10cm}

\vspace{3mm}

\checkbox \textbf{公共交通機関の運行状況を確認する}

\quad \circlecheck 正常運行 \quad \circlecheck 運行停止 \quad \circlecheck 情報収集中

\vspace{3mm}

\checkbox \textbf{緊急車両通行ルート・指定優先道路の状況を把握する}

\quad 緊急車両通行:\circlecheck 可能 \circlecheck 困難 \circlecheck 不可

\vspace{3mm}

\checkbox \textbf{透析液供給システムの損傷を確認する}

\quad \circlecheck 正常 \quad \circlecheck 一部損傷 \quad \circlecheck 大幅損傷

\vspace{5mm}

\section*{4. 緊急時連絡先}

\subsection*{4-1. 報告先}
\checkbox 福井県透析施設ネットワーク本部

\checkbox 福井県災害対策本部

\checkbox 日本透析医会災害情報ネットワーク

\checkbox 所在地市町村災害対策本部

\subsection*{4-2. 通信手段(優先順位)}
1. LINE・Teams

2. メーリングリスト

3. 衛星電話・MCA無線

4. 市町村経由報告

\vspace{5mm}

\noindent
\textbf{総合確認:} \\
\checkbox 患者・スタッフの安全確保完了 \\
\checkbox 施設設備点検完了 \\
\checkbox 初動報告(30分以内)完了 \\
\checkbox 次段階準備完了

\vspace{5mm}

\noindent
最終確認者:\underlinespace{4cm} \\
\vspace{3mm}
確認日時:\underlinespace{2cm}年\underlinespace{1cm}月\underlinespace{1cm}日 \quad \circlecheck 午前 \quad \circlecheck 午後 \quad \underlinespace{1cm}時\underlinespace{1cm}分

\end{document>