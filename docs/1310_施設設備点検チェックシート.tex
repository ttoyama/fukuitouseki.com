% Options for packages loaded elsewhere
% Options for packages loaded elsewhere
\PassOptionsToPackage{unicode}{hyperref}
\PassOptionsToPackage{hyphens}{url}
\PassOptionsToPackage{dvipsnames,svgnames,x11names}{xcolor}
%
\documentclass[
  japanese,
]{jarticle}
\usepackage{xcolor}
\usepackage[top=30mm,bottom=30mm,left=20mm,right=20mm,footskip=18mm,headsep=12mm]{geometry}
\usepackage{amsmath,amssymb}
\setcounter{secnumdepth}{-\maxdimen} % remove section numbering
\usepackage{iftex}
\ifPDFTeX
  \usepackage[T1]{fontenc}
  \usepackage[utf8]{inputenc}
  \usepackage{textcomp} % provide euro and other symbols
\else % if luatex or xetex
  \usepackage{unicode-math} % this also loads fontspec
  \defaultfontfeatures{Scale=MatchLowercase}
  \defaultfontfeatures[\rmfamily]{Ligatures=TeX,Scale=1}
\fi
\usepackage{lmodern}
\ifPDFTeX\else
  % xetex/luatex font selection
\fi
% Use upquote if available, for straight quotes in verbatim environments
\IfFileExists{upquote.sty}{\usepackage{upquote}}{}
\IfFileExists{microtype.sty}{% use microtype if available
  \usepackage[]{microtype}
  \UseMicrotypeSet[protrusion]{basicmath} % disable protrusion for tt fonts
}{}
\makeatletter
\@ifundefined{KOMAClassName}{% if non-KOMA class
  \IfFileExists{parskip.sty}{%
    \usepackage{parskip}
  }{% else
    \setlength{\parindent}{0pt}
    \setlength{\parskip}{6pt plus 2pt minus 1pt}}
}{% if KOMA class
  \KOMAoptions{parskip=half}}
\makeatother
% Make \paragraph and \subparagraph free-standing
\makeatletter
\ifx\paragraph\undefined\else
  \let\oldparagraph\paragraph
  \renewcommand{\paragraph}{
    \@ifstar
      \xxxParagraphStar
      \xxxParagraphNoStar
  }
  \newcommand{\xxxParagraphStar}[1]{\oldparagraph*{#1}\mbox{}}
  \newcommand{\xxxParagraphNoStar}[1]{\oldparagraph{#1}\mbox{}}
\fi
\ifx\subparagraph\undefined\else
  \let\oldsubparagraph\subparagraph
  \renewcommand{\subparagraph}{
    \@ifstar
      \xxxSubParagraphStar
      \xxxSubParagraphNoStar
  }
  \newcommand{\xxxSubParagraphStar}[1]{\oldsubparagraph*{#1}\mbox{}}
  \newcommand{\xxxSubParagraphNoStar}[1]{\oldsubparagraph{#1}\mbox{}}
\fi
\makeatother


\usepackage{longtable,booktabs,array}
\usepackage{calc} % for calculating minipage widths
% Correct order of tables after \paragraph or \subparagraph
\usepackage{etoolbox}
\makeatletter
\patchcmd\longtable{\par}{\if@noskipsec\mbox{}\fi\par}{}{}
\makeatother
% Allow footnotes in longtable head/foot
\IfFileExists{footnotehyper.sty}{\usepackage{footnotehyper}}{\usepackage{footnote}}
\makesavenoteenv{longtable}
\usepackage{graphicx}
\makeatletter
\newsavebox\pandoc@box
\newcommand*\pandocbounded[1]{% scales image to fit in text height/width
  \sbox\pandoc@box{#1}%
  \Gscale@div\@tempa{\textheight}{\dimexpr\ht\pandoc@box+\dp\pandoc@box\relax}%
  \Gscale@div\@tempb{\linewidth}{\wd\pandoc@box}%
  \ifdim\@tempb\p@<\@tempa\p@\let\@tempa\@tempb\fi% select the smaller of both
  \ifdim\@tempa\p@<\p@\scalebox{\@tempa}{\usebox\pandoc@box}%
  \else\usebox{\pandoc@box}%
  \fi%
}
% Set default figure placement to htbp
\def\fps@figure{htbp}
\makeatother



\ifLuaTeX
\usepackage[bidi=basic,provide=*]{babel}
\else
\usepackage[bidi=default,provide=*]{babel}
\fi
% get rid of language-specific shorthands (see #6817):
\let\LanguageShortHands\languageshorthands
\def\languageshorthands#1{}


\setlength{\emergencystretch}{3em} % prevent overfull lines

\providecommand{\tightlist}{%
  \setlength{\itemsep}{0pt}\setlength{\parskip}{0pt}}



 


\usepackage{setspace}
\setstretch{1.3}
\usepackage{array}
\usepackage{longtable}
\usepackage{amssymb}
\usepackage{multirow}
\usepackage{booktabs}
\newcommand{\checkbox}{$\square$\ }
\newcommand{\checkedbox}{$\blacksquare$\ }
\newcommand{\underlinespace}[1]{\underline{\hspace{#1}}}
\newcommand{\circlecheck}{$\bigcirc$\ }
\makeatletter
\@ifpackageloaded{caption}{}{\usepackage{caption}}
\AtBeginDocument{%
\ifdefined\contentsname
  \renewcommand*\contentsname{目次}
\else
  \newcommand\contentsname{目次}
\fi
\ifdefined\listfigurename
  \renewcommand*\listfigurename{図一覧}
\else
  \newcommand\listfigurename{図一覧}
\fi
\ifdefined\listtablename
  \renewcommand*\listtablename{表一覧}
\else
  \newcommand\listtablename{表一覧}
\fi
\ifdefined\figurename
  \renewcommand*\figurename{図}
\else
  \newcommand\figurename{図}
\fi
\ifdefined\tablename
  \renewcommand*\tablename{表}
\else
  \newcommand\tablename{表}
\fi
}
\@ifpackageloaded{float}{}{\usepackage{float}}
\floatstyle{ruled}
\@ifundefined{c@chapter}{\newfloat{codelisting}{h}{lop}}{\newfloat{codelisting}{h}{lop}[chapter]}
\floatname{codelisting}{コード}
\newcommand*\listoflistings{\listof{codelisting}{コード一覧}}
\makeatother
\makeatletter
\makeatother
\makeatletter
\@ifpackageloaded{caption}{}{\usepackage{caption}}
\@ifpackageloaded{subcaption}{}{\usepackage{subcaption}}
\makeatother
\usepackage{bookmark}
\IfFileExists{xurl.sty}{\usepackage{xurl}}{} % add URL line breaks if available
\urlstyle{same}
\hypersetup{
  pdftitle={施設設備点検チェックシート(平時の準備)},
  pdflang={ja},
  colorlinks=true,
  linkcolor={blue},
  filecolor={Maroon},
  citecolor={Blue},
  urlcolor={Blue},
  pdfcreator={LaTeX via pandoc}}


\title{施設設備点検チェックシート(平時の準備)}
\usepackage{etoolbox}
\makeatletter
\providecommand{\subtitle}[1]{% add subtitle to \maketitle
  \apptocmd{\@title}{\par {\large #1 \par}}{}{}
}
\makeatother
\subtitle{211 施設設備点検チェックシート(平時の準備)}
\author{}
\date{2025-09-08}
\begin{document}
\maketitle


\% 手動でタイトルを作成

\section{施設設備点検チェックシート(平時の準備)}\label{ux65bdux8a2dux8a2dux5099ux70b9ux691cux30c1ux30a7ux30c3ux30afux30b7ux30fcux30c8ux5e73ux6642ux306eux6e96ux5099}

\textbf{施設名:} \underlinespace{8cm}

\textbf{点検実施者:} \underlinespace{4cm} \textbf{点検日:}
\underlinespace{3cm}

\textbf{点検責任者:} \underlinespace{4cm} \textbf{確認日:}
\underlinespace{3cm}

\% 基本的な設備点検

\textbf{\large 基本設備の定期点検}

\checkbox 建物の構造的安全性:\checkbox 問題なし \checkbox 要注意箇所有
\checkbox 要修繕

要注意・修繕箇所:\underlinespace{10cm}

\checkbox 防災機能の点検:\checkbox 正常 \checkbox 一部不良
\checkbox 要修理

(\checkbox 火災警報 \checkbox スプリンクラー \checkbox エレベーター
\checkbox その他)

\checkbox 避難経路の確保:\checkbox 確保済 \checkbox 一部阻害
\checkbox 要改善

問題箇所:\underlinespace{10cm}

\% ライフライン点検

\textbf{\large ライフラインの点検と確保}

\checkbox 電力設備:\checkbox 正常 \checkbox 要点検 \checkbox 要修繕

\checkbox 自家発電装置:\checkbox 有・正常 \checkbox 有・要点検
\checkbox 無

稼働可能時間:\underlinespace{3cm} 燃料備蓄量:\underlinespace{3cm}

\checkbox 上水道設備:\checkbox 正常 \checkbox 要点検 \checkbox 要修繕

\checkbox 貯水設備:\checkbox 有(\underlinespace{2cm}m\(^3\))
\checkbox 無

\checkbox ガス設備:\checkbox 正常 \checkbox 要点検 \checkbox 要修繕
\checkbox 無

\checkbox 通信設備:\checkbox 正常 \checkbox 要点検 \checkbox 要修繕

(\checkbox 固定電話 \checkbox 携帯電話 \checkbox インターネット
\checkbox その他)

\% 透析関連設備点検

\textbf{\large 透析関連設備の点検}

\checkbox 透析液作成装置:\checkbox 固定済・正常 \checkbox 要固定
\checkbox 要点検

アンカーボルト固定:\checkbox 実施済 \checkbox 未実施

\checkbox 透析用監視装置(コンソール):\checkbox 正常 \checkbox 要点検

耐震設定:\checkbox 耐震構造(キャスターフリー)
\checkbox 免振構造(キャスターロック)

\checkbox 透析ベッド:\checkbox 正常 \checkbox 要点検

コンソール連結:\checkbox 実施済(バックルベルト等) \checkbox 未実施

\checkbox 透析用給水配管:\checkbox 正常 \checkbox 要点検
\checkbox 要改善

フレキシブル管対応:\checkbox 実施済 \checkbox 一部実施 \checkbox 未実施

\checkbox 水処理装置:\checkbox 正常 \checkbox 要点検 \checkbox 要修繕

\% 緊急対応物品

\textbf{\large 緊急対応物品の整備}

\begin{longtable*}{|p{4cm}|p{2cm}|p{2cm}|p{3cm}|}
\hline
**物品名** & **配置場所** & **動作確認** & **備考** \\
\hline
停電用懐中電灯 & \checkbox 配置済 & \checkbox 良好 & \\[0.5cm]
\hline
携帯テレビ・ラジオ & \checkbox 配置済 & \checkbox 良好 & \\[0.5cm]
\hline
患者誘導用ハンドマイク & \checkbox 配置済 & \checkbox 良好 & \\[0.5cm]
\hline
救急処置物品 & \checkbox 配置済 & \checkbox 良好 & \\[0.5cm]
\hline
救急カート & \checkbox 配置済 & \checkbox 良好 & \\[0.5cm]
\hline
AED & \checkbox 配置済 & \checkbox 良好 & \\[0.5cm]
\hline
血圧計 & \checkbox 配置済 & \checkbox 良好 & \\[0.5cm]
\hline
\end{longtable*}

\% 業者連絡先確認

\textbf{\large 関係業者連絡先の確認}

\checkbox 電力会社との連絡体制:\checkbox 確認済 \checkbox 要確認

\checkbox 水道事務所との連絡体制:\checkbox 確認済 \checkbox 要確認

\checkbox ガス会社との連絡体制:\checkbox 確認済 \checkbox 要確認

\checkbox 取引先業者との調整:\checkbox 確認済 \checkbox 要確認

\checkbox ビル管理会社との調整:\checkbox 確認済 \checkbox 要確認
\checkbox 該当なし

\% 改善事項・次回点検予定

\textbf{\large 改善事項・次回点検予定}

\textbf{要改善事項:}

\begin{enumerate}
\def\labelenumi{\arabic{enumi}.}
\item
  \underlinespace{12cm}
\item
  \underlinespace{12cm}
\item
  \underlinespace{12cm}
\end{enumerate}

\textbf{次回点検予定日:} \underlinespace{4cm} \textbf{実施予定者:}
\underlinespace{4cm}

\textbf{点検責任者確認:} \underlinespace{4cm} \textbf{確認印:}
\underlinespace{3cm}




\end{document}
