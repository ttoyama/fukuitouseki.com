% Options for packages loaded elsewhere
% Options for packages loaded elsewhere
\PassOptionsToPackage{unicode}{hyperref}
\PassOptionsToPackage{hyphens}{url}
\PassOptionsToPackage{dvipsnames,svgnames,x11names}{xcolor}
%
\documentclass[
  japanese,
]{jarticle}
\usepackage{xcolor}
\usepackage[top=30mm,bottom=30mm,left=20mm,right=20mm,footskip=18mm,headsep=12mm]{geometry}
\usepackage{amsmath,amssymb}
\setcounter{secnumdepth}{-\maxdimen} % remove section numbering
\usepackage{iftex}
\ifPDFTeX
  \usepackage[T1]{fontenc}
  \usepackage[utf8]{inputenc}
  \usepackage{textcomp} % provide euro and other symbols
\else % if luatex or xetex
  \usepackage{unicode-math} % this also loads fontspec
  \defaultfontfeatures{Scale=MatchLowercase}
  \defaultfontfeatures[\rmfamily]{Ligatures=TeX,Scale=1}
\fi
\usepackage{lmodern}
\ifPDFTeX\else
  % xetex/luatex font selection
\fi
% Use upquote if available, for straight quotes in verbatim environments
\IfFileExists{upquote.sty}{\usepackage{upquote}}{}
\IfFileExists{microtype.sty}{% use microtype if available
  \usepackage[]{microtype}
  \UseMicrotypeSet[protrusion]{basicmath} % disable protrusion for tt fonts
}{}
\makeatletter
\@ifundefined{KOMAClassName}{% if non-KOMA class
  \IfFileExists{parskip.sty}{%
    \usepackage{parskip}
  }{% else
    \setlength{\parindent}{0pt}
    \setlength{\parskip}{6pt plus 2pt minus 1pt}}
}{% if KOMA class
  \KOMAoptions{parskip=half}}
\makeatother
% Make \paragraph and \subparagraph free-standing
\makeatletter
\ifx\paragraph\undefined\else
  \let\oldparagraph\paragraph
  \renewcommand{\paragraph}{
    \@ifstar
      \xxxParagraphStar
      \xxxParagraphNoStar
  }
  \newcommand{\xxxParagraphStar}[1]{\oldparagraph*{#1}\mbox{}}
  \newcommand{\xxxParagraphNoStar}[1]{\oldparagraph{#1}\mbox{}}
\fi
\ifx\subparagraph\undefined\else
  \let\oldsubparagraph\subparagraph
  \renewcommand{\subparagraph}{
    \@ifstar
      \xxxSubParagraphStar
      \xxxSubParagraphNoStar
  }
  \newcommand{\xxxSubParagraphStar}[1]{\oldsubparagraph*{#1}\mbox{}}
  \newcommand{\xxxSubParagraphNoStar}[1]{\oldsubparagraph{#1}\mbox{}}
\fi
\makeatother


\usepackage{longtable,booktabs,array}
\usepackage{calc} % for calculating minipage widths
% Correct order of tables after \paragraph or \subparagraph
\usepackage{etoolbox}
\makeatletter
\patchcmd\longtable{\par}{\if@noskipsec\mbox{}\fi\par}{}{}
\makeatother
% Allow footnotes in longtable head/foot
\IfFileExists{footnotehyper.sty}{\usepackage{footnotehyper}}{\usepackage{footnote}}
\makesavenoteenv{longtable}
\usepackage{graphicx}
\makeatletter
\newsavebox\pandoc@box
\newcommand*\pandocbounded[1]{% scales image to fit in text height/width
  \sbox\pandoc@box{#1}%
  \Gscale@div\@tempa{\textheight}{\dimexpr\ht\pandoc@box+\dp\pandoc@box\relax}%
  \Gscale@div\@tempb{\linewidth}{\wd\pandoc@box}%
  \ifdim\@tempb\p@<\@tempa\p@\let\@tempa\@tempb\fi% select the smaller of both
  \ifdim\@tempa\p@<\p@\scalebox{\@tempa}{\usebox\pandoc@box}%
  \else\usebox{\pandoc@box}%
  \fi%
}
% Set default figure placement to htbp
\def\fps@figure{htbp}
\makeatother



\ifLuaTeX
\usepackage[bidi=basic,provide=*]{babel}
\else
\usepackage[bidi=default,provide=*]{babel}
\fi
% get rid of language-specific shorthands (see #6817):
\let\LanguageShortHands\languageshorthands
\def\languageshorthands#1{}


\setlength{\emergencystretch}{3em} % prevent overfull lines

\providecommand{\tightlist}{%
  \setlength{\itemsep}{0pt}\setlength{\parskip}{0pt}}



 


\usepackage{setspace}
\setstretch{1.3}
\usepackage{array}
\usepackage{longtable}
\usepackage{amssymb}
\usepackage{multirow}
\usepackage{booktabs}
\newcommand{\checkbox}{$\square$\ }
\newcommand{\checkedbox}{$\blacksquare$\ }
\newcommand{\underlinespace}[1]{\underline{\hspace{#1}}}
\newcommand{\circlecheck}{$\bigcirc$\ }
\makeatletter
\@ifpackageloaded{caption}{}{\usepackage{caption}}
\AtBeginDocument{%
\ifdefined\contentsname
  \renewcommand*\contentsname{目次}
\else
  \newcommand\contentsname{目次}
\fi
\ifdefined\listfigurename
  \renewcommand*\listfigurename{図一覧}
\else
  \newcommand\listfigurename{図一覧}
\fi
\ifdefined\listtablename
  \renewcommand*\listtablename{表一覧}
\else
  \newcommand\listtablename{表一覧}
\fi
\ifdefined\figurename
  \renewcommand*\figurename{図}
\else
  \newcommand\figurename{図}
\fi
\ifdefined\tablename
  \renewcommand*\tablename{表}
\else
  \newcommand\tablename{表}
\fi
}
\@ifpackageloaded{float}{}{\usepackage{float}}
\floatstyle{ruled}
\@ifundefined{c@chapter}{\newfloat{codelisting}{h}{lop}}{\newfloat{codelisting}{h}{lop}[chapter]}
\floatname{codelisting}{コード}
\newcommand*\listoflistings{\listof{codelisting}{コード一覧}}
\makeatother
\makeatletter
\makeatother
\makeatletter
\@ifpackageloaded{caption}{}{\usepackage{caption}}
\@ifpackageloaded{subcaption}{}{\usepackage{subcaption}}
\makeatother
\usepackage{bookmark}
\IfFileExists{xurl.sty}{\usepackage{xurl}}{} % add URL line breaks if available
\urlstyle{same}
\hypersetup{
  pdftitle={初動対応チェックシート(発災後0~30分)},
  pdflang={ja},
  colorlinks=true,
  linkcolor={blue},
  filecolor={Maroon},
  citecolor={Blue},
  urlcolor={Blue},
  pdfcreator={LaTeX via pandoc}}


\title{初動対応チェックシート(発災後0~30分)}
\usepackage{etoolbox}
\makeatletter
\providecommand{\subtitle}[1]{% add subtitle to \maketitle
  \apptocmd{\@title}{\par {\large #1 \par}}{}{}
}
\makeatother
\subtitle{101 初動対応チェックシート(発災後0~30分)}
\author{}
\date{2025-09-08}
\begin{document}
\maketitle


\% 手動でタイトルを作成

\section{初動対応チェックシート(発災後0~30分)}\label{ux521dux52d5ux5bfeux5fdcux30c1ux30a7ux30c3ux30afux30b7ux30fcux30c8ux767aux707dux5f8c030ux5206}

\textbf{施設名:} \underlinespace{8cm}

\textbf{記録者:} \underlinespace{4cm}

\textbf{発災日時:}
\underlinespace{2cm}年\underlinespace{1cm}月\underlinespace{1cm}日
\circlecheck 午前 \circlecheck 午後
\underlinespace{1cm}時\underlinespace{1cm}分

\subsection{1.
患者・スタッフ安全確保}\label{ux60a3ux8005ux30b9ux30bfux30c3ux30d5ux5b89ux5168ux78baux4fdd}

\subsubsection{1-1.
施設内患者安全確保}\label{ux65bdux8a2dux5185ux60a3ux8005ux5b89ux5168ux78baux4fdd}

\checkbox **施設内患者(透析中・待機中)の安全を確保する**

\begin{verbatim}
 確認時刻:\underlinespace{1cm}時\underlinespace{1cm}分      対応者:\underlinespace{4cm}
\end{verbatim}

\checkbox **負傷者の有無を確認し、応急処置を実施する**

\begin{verbatim}
 負傷者数:\underlinespace{2cm}人      処置内容:\underlinespace{6cm}
\end{verbatim}

\checkbox **パニック患者のそばに駆け寄り、心理的ケアを行う**

\begin{verbatim}
 対象患者数:\underlinespace{2cm}人      対応者:\underlinespace{4cm}
\end{verbatim}

\checkbox **透析中患者の治療中止・継続を判断する**

\begin{verbatim}
 透析中患者数:\underlinespace{2cm}人      継続:\underlinespace{2cm}人      中止:\underlinespace{2cm}人
\end{verbatim}

\subsubsection{1-2.
スタッフ安否確認}\label{ux30b9ux30bfux30c3ux30d5ux5b89ux5426ux78baux8a8d}

\checkbox **緊急連絡網を使用して全スタッフの安否を確認する**

\begin{verbatim}
 確認完了時刻:\underlinespace{1cm}時\underlinespace{1cm}分
\end{verbatim}

\checkbox **参集可能スタッフに速やかな施設参集を要請する**

\begin{verbatim}
 要請完了時刻:\underlinespace{1cm}時\underlinespace{1cm}分
\end{verbatim}

\subsection{2. 施設設備点検}\label{ux65bdux8a2dux8a2dux5099ux70b9ux691c}

\checkbox **建物本体の損傷を点検する**

\begin{verbatim}
 \circlecheck 安全      \circlecheck 要注意      \circlecheck 危険      点検者:\underlinespace{4cm}
\end{verbatim}

\checkbox **透析装置の稼働状況・警報発生の有無を確認する**

\begin{verbatim}
 稼働可能台数:\underlinespace{2cm}台/総\underlinespace{2cm}台      警報発生:\checkbox あり \checkbox なし
\end{verbatim}

\checkbox **水処理装置の異常を確認する**

\begin{verbatim}
 \circlecheck 正常      \circlecheck 異常あり(詳細:\underlinespace{6cm})
\end{verbatim}

\checkbox **自家発電設備の動作状況を点検する**

\begin{verbatim}
 \circlecheck 正常      \circlecheck 異常あり      \circlecheck 未確認      燃料残量:\underlinespace{3cm}
\end{verbatim}

\checkbox **医療ガス設備を確認する**

\begin{verbatim}
 \circlecheck 正常      \circlecheck 異常あり(詳細:\underlinespace{6cm})
\end{verbatim}

\checkbox **医薬品・医療材料の保管状況を確認する**

\begin{verbatim}
 \circlecheck 問題なし      \circlecheck 破損あり(詳細:\underlinespace{6cm})
\end{verbatim}

\checkbox **ライフライン状況を詳細に確認する**

\begin{verbatim}
 **電力:** \circlecheck 正常 \circlecheck 停電      **水道:** \circlecheck 正常 \circlecheck 断水

 **ガス:** \circlecheck 正常 \circlecheck 停止      **通信:** \circlecheck 正常 \circlecheck 不通
\end{verbatim}

\newpage

\subsection{3.
初動報告(発災後30分以内)}\label{ux521dux52d5ux5831ux544aux767aux707dux5f8c30ux5206ux4ee5ux5185}

\subsubsection{3-1.
福井県透析施設ネットワーク本部への報告}\label{ux798fux4e95ux770cux900fux6790ux65bdux8a2dux30cdux30c3ux30c8ux30efux30fcux30afux672cux90e8ux3078ux306eux5831ux544a}

\checkbox **施設職員・患者の安否確認結果を報告する**

\begin{verbatim}
 報告完了時刻:\underlinespace{1cm}時\underlinespace{1cm}分      報告者:\underlinespace{4cm}

 患者安否:安全\underlinespace{2cm}人、負傷\underlinespace{2cm}人、不明\underlinespace{2cm}人

 スタッフ安否:安全\underlinespace{2cm}人、負傷\underlinespace{2cm}人、不明\underlinespace{2cm}人
\end{verbatim}

\checkbox **建物の緊急危険度を判定して報告する**

\begin{verbatim}
 \circlecheck 安全(継続使用可能)      \circlecheck 要注意(一部制限)      \circlecheck 危険(使用不可)
\end{verbatim}

\checkbox **即座に必要な緊急支援の有無を報告する**

\begin{verbatim}
 \checkbox 緊急支援不要      \checkbox 緊急支援必要

 必要支援内容:\underlinespace{10cm}
\end{verbatim}

\subsubsection{3-2.
詳細状況確認(30分~1時間の準備)}\label{ux8a73ux7d30ux72b6ux6cc1ux78baux8a8d30ux52061ux6642ux9593ux306eux6e96ux5099}

\checkbox **施設周辺道路状況を確認する**

\begin{verbatim}
 主要アクセス路:\circlecheck 通行可能 \circlecheck 通行困難 \circlecheck 通行不可

 詳細:\underlinespace{10cm}
\end{verbatim}

\checkbox **公共交通機関の運行状況を確認する**

\begin{verbatim}
 \circlecheck 正常運行      \circlecheck 運行停止      \circlecheck 情報収集中
\end{verbatim}

\checkbox **緊急車両通行ルート・指定優先道路の状況を把握する**

\begin{verbatim}
 緊急車両通行:\circlecheck 可能 \circlecheck 困難 \circlecheck 不可
\end{verbatim}

\checkbox **透析液供給システムの損傷を確認する**

\begin{verbatim}
 \circlecheck 正常      \circlecheck 一部損傷      \circlecheck 大幅損傷
\end{verbatim}

\subsection{4. 緊急時連絡先}\label{ux7dcaux6025ux6642ux9023ux7d61ux5148}

\subsubsection{4-1. 報告先}\label{ux5831ux544aux5148}

\checkbox 福井県透析施設ネットワーク本部

\checkbox 福井県災害対策本部

\checkbox 日本透析医会災害情報ネットワーク

\checkbox 所在地市町村災害対策本部

\subsubsection{4-2.
通信手段(優先順位)}\label{ux901aux4fe1ux624bux6bb5ux512aux5148ux9806ux4f4d}

\begin{enumerate}
\def\labelenumi{\arabic{enumi}.}
\item
  LINE・Teams
\item
  メーリングリスト
\item
  衛星電話・MCA無線
\item
  市町村経由報告
\end{enumerate}

\textbf{総合確認:} \textbackslash{}
\checkbox 患者・スタッフの安全確保完了 \textbackslash{}
\checkbox 施設設備点検完了 \textbackslash{}
\checkbox 初動報告(30分以内)完了 \textbackslash{}
\checkbox 次段階準備完了

最終確認者:\underlinespace{4cm} \textbackslash{}

確認日時:\underlinespace{2cm}年\underlinespace{1cm}月\underlinespace{1cm}日
\circlecheck 午前 \circlecheck 午後
\underlinespace{1cm}時\underlinespace{1cm}分




\end{document}
