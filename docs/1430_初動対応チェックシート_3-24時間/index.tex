% Options for packages loaded elsewhere
% Options for packages loaded elsewhere
\PassOptionsToPackage{unicode}{hyperref}
\PassOptionsToPackage{hyphens}{url}
\PassOptionsToPackage{dvipsnames,svgnames,x11names}{xcolor}
%
\documentclass[
  japanese,
  letterpaper,
  DIV=11,
  numbers=noendperiod]{scrartcl}
\usepackage{xcolor}
\usepackage{amsmath,amssymb}
\setcounter{secnumdepth}{-\maxdimen} % remove section numbering
\usepackage{iftex}
\ifPDFTeX
  \usepackage[T1]{fontenc}
  \usepackage[utf8]{inputenc}
  \usepackage{textcomp} % provide euro and other symbols
\else % if luatex or xetex
  \usepackage{unicode-math} % this also loads fontspec
  \defaultfontfeatures{Scale=MatchLowercase}
  \defaultfontfeatures[\rmfamily]{Ligatures=TeX,Scale=1}
\fi
\usepackage{lmodern}
\ifPDFTeX\else
  % xetex/luatex font selection
\fi
% Use upquote if available, for straight quotes in verbatim environments
\IfFileExists{upquote.sty}{\usepackage{upquote}}{}
\IfFileExists{microtype.sty}{% use microtype if available
  \usepackage[]{microtype}
  \UseMicrotypeSet[protrusion]{basicmath} % disable protrusion for tt fonts
}{}
\makeatletter
\@ifundefined{KOMAClassName}{% if non-KOMA class
  \IfFileExists{parskip.sty}{%
    \usepackage{parskip}
  }{% else
    \setlength{\parindent}{0pt}
    \setlength{\parskip}{6pt plus 2pt minus 1pt}}
}{% if KOMA class
  \KOMAoptions{parskip=half}}
\makeatother
% Make \paragraph and \subparagraph free-standing
\makeatletter
\ifx\paragraph\undefined\else
  \let\oldparagraph\paragraph
  \renewcommand{\paragraph}{
    \@ifstar
      \xxxParagraphStar
      \xxxParagraphNoStar
  }
  \newcommand{\xxxParagraphStar}[1]{\oldparagraph*{#1}\mbox{}}
  \newcommand{\xxxParagraphNoStar}[1]{\oldparagraph{#1}\mbox{}}
\fi
\ifx\subparagraph\undefined\else
  \let\oldsubparagraph\subparagraph
  \renewcommand{\subparagraph}{
    \@ifstar
      \xxxSubParagraphStar
      \xxxSubParagraphNoStar
  }
  \newcommand{\xxxSubParagraphStar}[1]{\oldsubparagraph*{#1}\mbox{}}
  \newcommand{\xxxSubParagraphNoStar}[1]{\oldsubparagraph{#1}\mbox{}}
\fi
\makeatother


\usepackage{longtable,booktabs,array}
\usepackage{calc} % for calculating minipage widths
% Correct order of tables after \paragraph or \subparagraph
\usepackage{etoolbox}
\makeatletter
\patchcmd\longtable{\par}{\if@noskipsec\mbox{}\fi\par}{}{}
\makeatother
% Allow footnotes in longtable head/foot
\IfFileExists{footnotehyper.sty}{\usepackage{footnotehyper}}{\usepackage{footnote}}
\makesavenoteenv{longtable}
\usepackage{graphicx}
\makeatletter
\newsavebox\pandoc@box
\newcommand*\pandocbounded[1]{% scales image to fit in text height/width
  \sbox\pandoc@box{#1}%
  \Gscale@div\@tempa{\textheight}{\dimexpr\ht\pandoc@box+\dp\pandoc@box\relax}%
  \Gscale@div\@tempb{\linewidth}{\wd\pandoc@box}%
  \ifdim\@tempb\p@<\@tempa\p@\let\@tempa\@tempb\fi% select the smaller of both
  \ifdim\@tempa\p@<\p@\scalebox{\@tempa}{\usebox\pandoc@box}%
  \else\usebox{\pandoc@box}%
  \fi%
}
% Set default figure placement to htbp
\def\fps@figure{htbp}
\makeatother



\ifLuaTeX
\usepackage[bidi=basic,provide=*]{babel}
\else
\usepackage[bidi=default,provide=*]{babel}
\fi
% get rid of language-specific shorthands (see #6817):
\let\LanguageShortHands\languageshorthands
\def\languageshorthands#1{}


\setlength{\emergencystretch}{3em} % prevent overfull lines

\providecommand{\tightlist}{%
  \setlength{\itemsep}{0pt}\setlength{\parskip}{0pt}}



 


% カスタムヘッダー設定(既存PDFと同じ)
\makeatletter
\def\@oddhead{\hfill\small 103 初動対応チェックシート(3-24時間) 2025.09.04版}
\def\@evenhead{\hfill\small 103 初動対応チェックシート(3-24時間) 2025.09.04版}
\def\@oddfoot{\hfil\thepage\hfil}
\def\@evenfoot{\hfil\thepage\hfil}
\makeatother
\KOMAoption{captions}{tableheading}
\makeatletter
\@ifpackageloaded{caption}{}{\usepackage{caption}}
\AtBeginDocument{%
\ifdefined\contentsname
  \renewcommand*\contentsname{目次}
\else
  \newcommand\contentsname{目次}
\fi
\ifdefined\listfigurename
  \renewcommand*\listfigurename{図一覧}
\else
  \newcommand\listfigurename{図一覧}
\fi
\ifdefined\listtablename
  \renewcommand*\listtablename{表一覧}
\else
  \newcommand\listtablename{表一覧}
\fi
\ifdefined\figurename
  \renewcommand*\figurename{図}
\else
  \newcommand\figurename{図}
\fi
\ifdefined\tablename
  \renewcommand*\tablename{表}
\else
  \newcommand\tablename{表}
\fi
}
\@ifpackageloaded{float}{}{\usepackage{float}}
\floatstyle{ruled}
\@ifundefined{c@chapter}{\newfloat{codelisting}{h}{lop}}{\newfloat{codelisting}{h}{lop}[chapter]}
\floatname{codelisting}{コード}
\newcommand*\listoflistings{\listof{codelisting}{コード一覧}}
\makeatother
\makeatletter
\makeatother
\makeatletter
\@ifpackageloaded{caption}{}{\usepackage{caption}}
\@ifpackageloaded{subcaption}{}{\usepackage{subcaption}}
\makeatother
\usepackage{bookmark}
\IfFileExists{xurl.sty}{\usepackage{xurl}}{} % add URL line breaks if available
\urlstyle{same}
\hypersetup{
  pdftitle={初動対応チェックシート(発災後3~24時間)},
  pdflang={ja-JP},
  colorlinks=true,
  linkcolor={blue},
  filecolor={Maroon},
  citecolor={Blue},
  urlcolor={Blue},
  pdfcreator={LaTeX via pandoc}}


\title{初動対応チェックシート(発災後3~24時間)}
\author{}
\date{}
\begin{document}
\maketitle


\begin{center}
{\Large\textbf{初動対応チェックシート(発災後3~24時間)}}
\end{center}
\vspace{5mm}

\noindent \textbf{施設名:} \underlinespace{8cm}

\vspace{3mm}

\noindent \textbf{記録者:} \underlinespace{4cm}

\vspace{3mm}

\noindent \textbf{記録日時:}
\underlinespace{2cm}年\underlinespace{1cm}月\underlinespace{1cm}日
\quad \circlecheck 午前 \quad \circlecheck 午後
\quad \underlinespace{1cm}時\underlinespace{1cm}分

\vspace{5mm}

\subsection{1. 詳細状況分析}\label{ux8a73ux7d30ux72b6ux6cc1ux5206ux6790}

\subsubsection{1-1.
物資・薬剤備蓄状況確認}\label{ux7269ux8cc7ux85acux5264ux5099ux84c4ux72b6ux6cc1ux78baux8a8d}

\checkbox **透析液(HD/PD)の残量を確認する**

\quad HD透析液残量:\underlinespace{3cm}L
\quad PD透析液残量:\underlinespace{3cm}L

\vspace{3mm}

\checkbox **回路・ダイアライザー・穿刺針の在庫を確認する**

\quad 回路:\underlinespace{3cm}本
\quad ダイアライザー:\underlinespace{3cm}本
\quad 穿刺針:\underlinespace{3cm}本

\vspace{3mm}

\checkbox **消毒液・常用薬・点滴薬を確認する**

\quad 消毒液:\underlinespace{3cm}L \quad 常用薬:\underlinespace{6cm}
\quad 点滴薬:\underlinespace{6cm}

\vspace{3mm}

\checkbox **食料・水の備蓄を確認する**

\quad 食料(人日分):\underlinespace{3cm}
\quad 飲料水(L):\underlinespace{3cm}

\vspace{5mm}

\subsubsection{1-2.
非常用電源稼働時間計算}\label{ux975eux5e38ux7528ux96fbux6e90ux7a3cux50cdux6642ux9593ux8a08ux7b97}

\checkbox **燃料残量を確認する**

\quad 燃料残量:\underlinespace{4cm}
\quad 燃料種別:\underlinespace{4cm}

\vspace{3mm}

\checkbox **透析装置の消費電力を計算する**

\quad 稼働台数:\underlinespace{2cm}台
\quad 総消費電力:\underlinespace{4cm}kW

\vspace{3mm}

\checkbox **透析継続可能時間を算出する**

\quad 継続可能時間:約\underlinespace{3cm}時間(\underlinespace{2cm}日分)

\newpage

\subsubsection{1-3.
ライフライン復旧見込み確認}\label{ux30e9ux30a4ux30d5ux30e9ux30a4ux30f3ux5fa9ux65e7ux898bux8fbcux307fux78baux8a8d}

\checkbox **ライフラインの復旧見込みを確認する**

\quad 電力復旧見込み:\underlinespace{2cm}日後
\quad 水道復旧見込み:\underlinespace{2cm}日後

\quad ガス復旧見込み:\underlinespace{2cm}日後
\quad 通信復旧見込み:\underlinespace{2cm}日後

\vspace{10mm}

\subsection{2. 患者対応準備}\label{ux60a3ux8005ux5bfeux5fdcux6e96ux5099}

\subsubsection{2-1.
患者カテゴリー別優先度設定}\label{ux60a3ux8005ux30abux30c6ux30b4ux30eaux30fcux5225ux512aux5148ux5ea6ux8a2dux5b9a}

\checkbox **緊急性の高い患者を識別する**

\quad 高K血症:\underlinespace{2cm}人
\quad 肺水腫:\underlinespace{2cm}人
\quad その他緊急:\underlinespace{2cm}人

\vspace{3mm}

\checkbox **最終透析からの経過時間により分類する**

\quad 96時間以上:\underlinespace{2cm}人
\quad 72-96時間:\underlinespace{2cm}人
\quad 48-72時間:\underlinespace{2cm}人

\vspace{3mm}

\checkbox **残存腎機能・併存疾患により分類する**

\quad 残存腎機能なし:\underlinespace{2cm}人
\quad 重篤併存疾患:\underlinespace{2cm}人

\vspace{3mm}

\checkbox **トリアージの実施準備をする**

\quad 最優先(赤):\underlinespace{2cm}人
\quad 優先(黄):\underlinespace{2cm}人
\quad 待機可能(緑):\underlinespace{2cm}人

\vspace{5mm}

\subsubsection{2-2.
患者への情報提供準備}\label{ux60a3ux8005ux3078ux306eux60c5ux5831ux63d0ux4f9bux6e96ux5099}

\checkbox **透析継続可否の説明資料を作成する**

\quad 作成完了時刻:\underlinespace{1cm}時\underlinespace{1cm}分
\quad 作成者:\underlinespace{4cm}

\vspace{3mm}

\checkbox **今後の見通し情報を整理する**

\quad 情報整理完了時刻:\underlinespace{1cm}時\underlinespace{1cm}分

\vspace{5mm}

\subsection{3.
アクセス確保措置}\label{ux30a2ux30afux30bbux30b9ux78baux4fddux63aaux7f6e}

\checkbox **通行規制情報を収集し共有する**

\quad 情報収集完了時刻:\underlinespace{1cm}時\underlinespace{1cm}分
\quad 情報源:\underlinespace{6cm}

\vspace{3mm}

\checkbox **道路啓開要請の準備をする(福井県庁経由)**

\quad 要請準備完了:\checkbox 済 \quad 要請実施:\checkbox 済
\quad 実施時刻:\underlinespace{1cm}時\underlinespace{1cm}分

\vspace{3mm}

\checkbox **緊急車両通行許可を申請する**

\quad 申請完了:\checkbox 済 \quad 許可取得:\checkbox 済
\quad 許可番号:\underlinespace{6cm}

\newpage

\checkbox **自衛隊等による搬送・輸送要請を検討する**

\quad 検討結果:\checkbox 要請 \checkbox 不要
\quad 要請内容:\underlinespace{8cm}

\vspace{3mm}

\checkbox **民間協力を要請する(建設業者・運送業者)**

\quad 要請先:\underlinespace{8cm} \quad 協力内容:\underlinespace{6cm}

\vspace{3mm}

\checkbox **ヘリコプター輸送の可能性を検討する**

\quad 検討結果:\checkbox 利用可能 \checkbox 利用困難
\quad ヘリポート:\underlinespace{6cm}

\vspace{10mm}

\subsection{4.
長期対応準備報告(発災後24時間)}\label{ux9577ux671fux5bfeux5fdcux6e96ux5099ux5831ux544aux767aux707dux5f8c24ux6642ux9593}

\subsubsection{4-1.
福井県透析施設ネットワーク本部への報告}\label{ux798fux4e95ux770cux900fux6790ux65bdux8a2dux30cdux30c3ux30c8ux30efux30fcux30afux672cux90e8ux3078ux306eux5831ux544a}

\checkbox **設備復旧の具体的見込みを報告する**

\quad 報告完了時刻:\underlinespace{1cm}時\underlinespace{1cm}分
\quad 報告者:\underlinespace{4cm}

\quad 完全復旧見込み:\underlinespace{2cm}日後
\quad 部分運営開始見込み:\underlinespace{2cm}日後

\vspace{3mm}

\checkbox **長期受け入れ困難患者数を報告する**

\quad 受入困難患者総数:\underlinespace{2cm}人

\quad 内訳:高緊急度\underlinespace{2cm}人、中緊急度\underlinespace{2cm}人、低緊急度\underlinespace{2cm}人

\vspace{3mm}

\checkbox **継続的な支援要請内容を報告する**

\checkbox 継続的人員派遣 \quad \checkbox 継続的物資支援
\quad \checkbox 継続的患者搬送

\quad 支援期間見込み:\underlinespace{4cm}
\quad 詳細要請内容:\underlinespace{8cm}

\vspace{3mm}

\checkbox **職員体制と業務継続計画を報告する**

\quad 職員参集率:\underlinespace{2cm}\%
\quad 業務継続レベル:\underlinespace{6cm}

\vspace{5mm}

\subsection{5.
継続的記録・情報管理}\label{ux7d99ux7d9aux7684ux8a18ux9332ux60c5ux5831ux7ba1ux7406}

\checkbox **継続的に対応記録を更新する**

\quad 記録更新間隔:\underlinespace{2cm}時間毎
\quad 記録責任者:\underlinespace{4cm}

\vspace{3mm}

\checkbox **複数の通信手段を組み合わせて確保する**

\quad 主通信手段:\underlinespace{6cm}
\quad 副通信手段:\underlinespace{6cm}

\vspace{3mm}

\checkbox **通信途絶時の市町村経由報告を準備する**

\quad 市町村連絡先:\underlinespace{8cm}
\quad 報告様式:\underlinespace{4cm}

\newpage

\subsection{6.
患者優先度の基準}\label{ux60a3ux8005ux512aux5148ux5ea6ux306eux57faux6e96}

\textbf{最優先(赤)}

\checkbox 高カリウム血症、肺水腫、呼吸困難

\checkbox 意識レベルの変化がある患者

\checkbox 最終透析から96時間以上経過

\vspace{5mm}

\textbf{優先(黄)}

\checkbox 残存腎機能がほとんどない患者

\checkbox 重篤な併存疾患のある患者

\vspace{5mm}

\textbf{待機可能(緑)}

\checkbox 腹膜透析(PD)で自己管理可能な患者

\checkbox 残存腎機能がある患者

\vspace{5mm}

\textbf{総合確認:}

\checkbox 詳細状況分析完了

\checkbox 患者対応準備完了

\checkbox アクセス確保措置完了

\checkbox 長期対応準備報告(24時間)完了

\checkbox 継続的記録・情報管理体制確立完了

\vspace{5mm}

\textbf{最終確認者:} \underlinespace{4cm}

\textbf{確認日時:}
\underlinespace{2cm}年\underlinespace{1cm}月\underlinespace{1cm}日
\quad \circlecheck 午前 \quad \circlecheck 午後
\quad \underlinespace{1cm}時\underlinespace{1cm}分




\end{document}
