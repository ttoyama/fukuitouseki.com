% 初動対応チェックシート(発災後3~24時間)
% LaTeX template for disaster dialysis manual forms

\documentclass[a4paper,12pt]{jarticle}
\usepackage[top=30mm, bottom=30mm, left=20mm, right=20mm, footskip=18mm, headsep=12mm]{geometry}
\usepackage{setspace}
\setstretch{1.3}
\usepackage{array}
\usepackage{longtable}
\usepackage{amssymb}
\usepackage{multirow}
\usepackage{booktabs}

\newcommand{\checkbox}{$\square$\ }
\newcommand{\checkedbox}{$\blacksquare$\ }
\newcommand{\underlinespace}[1]{\underline{\hspace{#1}}}
\newcommand{\circlecheck}{$\bigcirc$\ }

% シンプルなページ番号設定
\pagestyle{plain}
\makeatletter
\def\@oddhead{\hfill\small 103 初動対応チェックシート(3-24時間) 2025.09.04版}
\def\@evenhead{\hfill\small 103 初動対応チェックシート(3-24時間) 2025.09.04版}
\def\@oddfoot{\hfil\thepage\hfil}
\def\@evenfoot{\hfil\thepage\hfil}
\makeatother

\begin{document}

% 手動でタイトルを作成
\begin{center}
{\Large\textbf{初動対応チェックシート(発災後3~24時間)}}
\end{center}
\vspace{5mm}

\noindent
\textbf{施設名:} \underlinespace{8cm}

\vspace{3mm}

\noindent
\textbf{記録者:} \underlinespace{4cm}

\vspace{3mm}

\noindent
\textbf{記録日時:} \underlinespace{2cm}年\underlinespace{1cm}月\underlinespace{1cm}日 \quad \circlecheck 午前 \quad \circlecheck 午後 \quad \underlinespace{1cm}時\underlinespace{1cm}分

\vspace{5mm}

\section*{1. 詳細状況分析}

\subsection*{1-1. 物資・薬剤備蓄状況確認}

\checkbox \textbf{透析液(HD/PD)の残量を確認する}

\quad HD透析液残量:\underlinespace{3cm}L \quad PD透析液残量:\underlinespace{3cm}L

\vspace{3mm}

\checkbox \textbf{回路・ダイアライザー・穿刺針の在庫を確認する}

\quad 回路:\underlinespace{3cm}本 \quad ダイアライザー:\underlinespace{3cm}本 \quad 穿刺針:\underlinespace{3cm}本

\vspace{3mm}

\checkbox \textbf{消毒液・常用薬・点滴薬を確認する}

\quad 消毒液:\underlinespace{3cm}L \quad 常用薬:\underlinespace{6cm} \quad 点滴薬:\underlinespace{6cm}

\vspace{3mm}

\checkbox \textbf{食料・水の備蓄を確認する}

\quad 食料(人日分):\underlinespace{3cm} \quad 飲料水(L):\underlinespace{3cm}

\vspace{5mm}

\subsection*{1-2. 非常用電源稼働時間計算}

\checkbox \textbf{燃料残量を確認する}

\quad 燃料残量:\underlinespace{4cm} \quad 燃料種別:\underlinespace{4cm}

\vspace{3mm}

\checkbox \textbf{透析装置の消費電力を計算する}

\quad 稼働台数:\underlinespace{2cm}台 \quad 総消費電力:\underlinespace{4cm}kW

\vspace{3mm}

\checkbox \textbf{透析継続可能時間を算出する}

\quad 継続可能時間:約\underlinespace{3cm}時間(\underlinespace{2cm}日分)

\vspace{5mm}

\subsection*{1-3. ライフライン復旧見込み確認}

\checkbox \textbf{ライフラインの復旧見込みを確認する}

\quad \textbf{電力復旧見込み:} \underlinespace{2cm}日後 \quad \textbf{水道復旧見込み:} \underlinespace{2cm}日後

\quad \textbf{ガス復旧見込み:} \underlinespace{2cm}日後 \quad \textbf{通信復旧見込み:} \underlinespace{2cm}日後

\vspace{5mm}

\newpage

\section*{2. 患者対応準備}

\subsection*{2-1. 患者カテゴリー別優先度設定}

\checkbox \textbf{緊急性の高い患者を識別する}

\quad 高K血症:\underlinespace{2cm}人 \quad 肺水腫:\underlinespace{2cm}人 \quad その他緊急:\underlinespace{2cm}人

\vspace{3mm}

\checkbox \textbf{最終透析からの経過時間により分類する}

\quad 96時間以上:\underlinespace{2cm}人 \quad 72-96時間:\underlinespace{2cm}人 \quad 48-72時間:\underlinespace{2cm}人

\vspace{3mm}

\checkbox \textbf{残存腎機能・併存疾患により分類する}

\quad 残存腎機能なし:\underlinespace{2cm}人 \quad 重篤併存疾患:\underlinespace{2cm}人

\vspace{3mm}

\checkbox \textbf{トリアージの実施準備をする}

\quad 最優先(赤):\underlinespace{2cm}人 \quad 優先(黄):\underlinespace{2cm}人 \quad 待機可能(緑):\underlinespace{2cm}人

\vspace{5mm}

\subsection*{2-2. 患者への情報提供準備}

\checkbox \textbf{透析継続可否の説明資料を作成する}

\quad 作成完了時刻:\underlinespace{1cm}時\underlinespace{1cm}分 \quad 作成者:\underlinespace{4cm}

\vspace{3mm}

\checkbox \textbf{今後の見通し情報を整理する}

\quad 情報整理完了時刻:\underlinespace{1cm}時\underlinespace{1cm}分

\vspace{5mm}

\section*{3. アクセス確保措置}

\checkbox \textbf{通行規制情報を収集し共有する}

\quad 情報収集完了時刻:\underlinespace{1cm}時\underlinespace{1cm}分 \quad 情報源:\underlinespace{6cm}

\vspace{3mm}

\checkbox \textbf{道路啓開要請の準備をする(福井県庁経由)}

\quad 要請準備完了:\checkbox 済 \quad 要請実施:\checkbox 済 \quad 実施時刻:\underlinespace{1cm}時\underlinespace{1cm}分

\vspace{3mm}

\checkbox \textbf{緊急車両通行許可を申請する}

\quad 申請完了:\checkbox 済 \quad 許可取得:\checkbox 済 \quad 許可番号:\underlinespace{6cm}

\vspace{3mm}

\checkbox \textbf{自衛隊等による搬送・輸送要請を検討する}

\quad 検討結果:\checkbox 要請 \checkbox 不要 \quad 要請内容:\underlinespace{8cm}

\vspace{3mm}

\checkbox \textbf{民間協力を要請する(建設業者・運送業者)}

\quad 要請先:\underlinespace{8cm} \quad 協力内容:\underlinespace{6cm}

\vspace{3mm}

\checkbox \textbf{ヘリコプター輸送の可能性を検討する}

\quad 検討結果:\checkbox 利用可能 \checkbox 利用困難 \quad ヘリポート:\underlinespace{6cm}

\vspace{5mm}

\newpage

\section*{4. 長期対応準備報告(発災後24時間)}

\subsection*{4-1. 福井県透析施設ネットワーク本部への報告}

\checkbox \textbf{設備復旧の具体的見込みを報告する}

\quad 報告完了時刻:\underlinespace{1cm}時\underlinespace{1cm}分 \quad 報告者:\underlinespace{4cm}

\quad 完全復旧見込み:\underlinespace{2cm}日後 \quad 部分運営開始見込み:\underlinespace{2cm}日後

\vspace{4mm}

\checkbox \textbf{長期受け入れ困難患者数を報告する}

\quad 受入困難患者総数:\underlinespace{2cm}人

\quad 内訳:高緊急度\underlinespace{2cm}人、中緊急度\underlinespace{2cm}人、低緊急度\underlinespace{2cm}人

\vspace{4mm}

\checkbox \textbf{継続的な支援要請内容を報告する}

\quad \checkbox 継続的人員派遣 \quad \checkbox 継続的物資支援 \quad \checkbox 継続的患者搬送

\quad 支援期間見込み:\underlinespace{4cm} \quad 詳細要請内容:\underlinespace{8cm}

\vspace{4mm}

\checkbox \textbf{職員体制と業務継続計画を報告する}

\quad 職員参集率:\underlinespace{2cm}\% \quad 業務継続レベル:\underlinespace{6cm}

\vspace{5mm}

\section*{5. 継続的記録・情報管理}

\checkbox \textbf{継続的に対応記録を更新する}

\quad 記録更新間隔:\underlinespace{2cm}時間毎 \quad 記録責任者:\underlinespace{4cm}

\vspace{3mm}

\checkbox \textbf{複数の通信手段を組み合わせて確保する}

\quad 主通信手段:\underlinespace{6cm} \quad 副通信手段:\underlinespace{6cm}

\vspace{3mm}

\checkbox \textbf{通信途絶時の市町村経由報告を準備する}

\quad 市町村連絡先:\underlinespace{8cm} \quad 報告様式:\underlinespace{4cm}

\vspace{5mm}

\section*{6. 患者優先度の基準}

\subsection*{最優先(赤)}
\checkbox 高カリウム血症、肺水腫、呼吸困難

\checkbox 意識レベルの変化がある患者

\checkbox 最終透析から96時間以上経過

\subsection*{優先(黄)}
\checkbox 残存腎機能がほとんどない患者

\checkbox 重篤な併存疾患のある患者

\subsection*{待機可能(緑)}
\checkbox 腹膜透析(PD)で自己管理可能な患者

\checkbox 残存腎機能がある患者

\vspace{5mm}

\noindent
\textbf{総合確認:} \\
\checkbox 詳細状況分析完了 \\
\checkbox 患者対応準備完了 \\
\checkbox アクセス確保措置完了 \\
\checkbox 長期対応準備報告(24時間)完了 \\
\checkbox 継続的記録・情報管理体制確立完了

\vspace{5mm}

\noindent
最終確認者:\underlinespace{4cm} \\
\vspace{3mm}
確認日時:\underlinespace{2cm}年\underlinespace{1cm}月\underlinespace{1cm}日 \quad \circlecheck 午前 \quad \circlecheck 午後 \quad \underlinespace{1cm}時\underlinespace{1cm}分

\end{document}