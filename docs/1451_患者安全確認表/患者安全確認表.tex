% Options for packages loaded elsewhere
% Options for packages loaded elsewhere
\PassOptionsToPackage{unicode}{hyperref}
\PassOptionsToPackage{hyphens}{url}
\PassOptionsToPackage{dvipsnames,svgnames,x11names}{xcolor}
%
\documentclass[
  japanese,
]{jarticle}
\usepackage{xcolor}
\usepackage[top=30mm,bottom=30mm,left=20mm,right=20mm,footskip=18mm,headsep=12mm]{geometry}
\usepackage{amsmath,amssymb}
\setcounter{secnumdepth}{-\maxdimen} % remove section numbering
\usepackage{iftex}
\ifPDFTeX
  \usepackage[T1]{fontenc}
  \usepackage[utf8]{inputenc}
  \usepackage{textcomp} % provide euro and other symbols
\else % if luatex or xetex
  \usepackage{unicode-math} % this also loads fontspec
  \defaultfontfeatures{Scale=MatchLowercase}
  \defaultfontfeatures[\rmfamily]{Ligatures=TeX,Scale=1}
\fi
\usepackage{lmodern}
\ifPDFTeX\else
  % xetex/luatex font selection
\fi
% Use upquote if available, for straight quotes in verbatim environments
\IfFileExists{upquote.sty}{\usepackage{upquote}}{}
\IfFileExists{microtype.sty}{% use microtype if available
  \usepackage[]{microtype}
  \UseMicrotypeSet[protrusion]{basicmath} % disable protrusion for tt fonts
}{}
\makeatletter
\@ifundefined{KOMAClassName}{% if non-KOMA class
  \IfFileExists{parskip.sty}{%
    \usepackage{parskip}
  }{% else
    \setlength{\parindent}{0pt}
    \setlength{\parskip}{6pt plus 2pt minus 1pt}}
}{% if KOMA class
  \KOMAoptions{parskip=half}}
\makeatother
% Make \paragraph and \subparagraph free-standing
\makeatletter
\ifx\paragraph\undefined\else
  \let\oldparagraph\paragraph
  \renewcommand{\paragraph}{
    \@ifstar
      \xxxParagraphStar
      \xxxParagraphNoStar
  }
  \newcommand{\xxxParagraphStar}[1]{\oldparagraph*{#1}\mbox{}}
  \newcommand{\xxxParagraphNoStar}[1]{\oldparagraph{#1}\mbox{}}
\fi
\ifx\subparagraph\undefined\else
  \let\oldsubparagraph\subparagraph
  \renewcommand{\subparagraph}{
    \@ifstar
      \xxxSubParagraphStar
      \xxxSubParagraphNoStar
  }
  \newcommand{\xxxSubParagraphStar}[1]{\oldsubparagraph*{#1}\mbox{}}
  \newcommand{\xxxSubParagraphNoStar}[1]{\oldsubparagraph{#1}\mbox{}}
\fi
\makeatother


\usepackage{longtable,booktabs,array}
\usepackage{calc} % for calculating minipage widths
% Correct order of tables after \paragraph or \subparagraph
\usepackage{etoolbox}
\makeatletter
\patchcmd\longtable{\par}{\if@noskipsec\mbox{}\fi\par}{}{}
\makeatother
% Allow footnotes in longtable head/foot
\IfFileExists{footnotehyper.sty}{\usepackage{footnotehyper}}{\usepackage{footnote}}
\makesavenoteenv{longtable}
\usepackage{graphicx}
\makeatletter
\newsavebox\pandoc@box
\newcommand*\pandocbounded[1]{% scales image to fit in text height/width
  \sbox\pandoc@box{#1}%
  \Gscale@div\@tempa{\textheight}{\dimexpr\ht\pandoc@box+\dp\pandoc@box\relax}%
  \Gscale@div\@tempb{\linewidth}{\wd\pandoc@box}%
  \ifdim\@tempb\p@<\@tempa\p@\let\@tempa\@tempb\fi% select the smaller of both
  \ifdim\@tempa\p@<\p@\scalebox{\@tempa}{\usebox\pandoc@box}%
  \else\usebox{\pandoc@box}%
  \fi%
}
% Set default figure placement to htbp
\def\fps@figure{htbp}
\makeatother



\ifLuaTeX
\usepackage[bidi=basic,provide=*]{babel}
\else
\usepackage[bidi=default,provide=*]{babel}
\fi
% get rid of language-specific shorthands (see #6817):
\let\LanguageShortHands\languageshorthands
\def\languageshorthands#1{}


\setlength{\emergencystretch}{3em} % prevent overfull lines

\providecommand{\tightlist}{%
  \setlength{\itemsep}{0pt}\setlength{\parskip}{0pt}}



 


\usepackage{setspace}
\setstretch{1.3}
\usepackage{array}
\usepackage{longtable}
\usepackage{amssymb}
\usepackage{multirow}
\usepackage{booktabs}
\newcommand{\checkbox}{$\square$\ }
\newcommand{\checkedbox}{$\blacksquare$\ }
\newcommand{\underlinespace}[1]{\underline{\hspace{#1}}}
\newcommand{\circlecheck}{$\bigcirc$\ }
\makeatletter
\@ifpackageloaded{caption}{}{\usepackage{caption}}
\AtBeginDocument{%
\ifdefined\contentsname
  \renewcommand*\contentsname{目次}
\else
  \newcommand\contentsname{目次}
\fi
\ifdefined\listfigurename
  \renewcommand*\listfigurename{図一覧}
\else
  \newcommand\listfigurename{図一覧}
\fi
\ifdefined\listtablename
  \renewcommand*\listtablename{表一覧}
\else
  \newcommand\listtablename{表一覧}
\fi
\ifdefined\figurename
  \renewcommand*\figurename{図}
\else
  \newcommand\figurename{図}
\fi
\ifdefined\tablename
  \renewcommand*\tablename{表}
\else
  \newcommand\tablename{表}
\fi
}
\@ifpackageloaded{float}{}{\usepackage{float}}
\floatstyle{ruled}
\@ifundefined{c@chapter}{\newfloat{codelisting}{h}{lop}}{\newfloat{codelisting}{h}{lop}[chapter]}
\floatname{codelisting}{コード}
\newcommand*\listoflistings{\listof{codelisting}{コード一覧}}
\makeatother
\makeatletter
\makeatother
\makeatletter
\@ifpackageloaded{caption}{}{\usepackage{caption}}
\@ifpackageloaded{subcaption}{}{\usepackage{subcaption}}
\makeatother
\usepackage{bookmark}
\IfFileExists{xurl.sty}{\usepackage{xurl}}{} % add URL line breaks if available
\urlstyle{same}
\hypersetup{
  pdftitle={様式04},
  pdflang={ja},
  colorlinks=true,
  linkcolor={blue},
  filecolor={Maroon},
  citecolor={Blue},
  urlcolor={Blue},
  pdfcreator={LaTeX via pandoc}}


\title{様式04}
\usepackage{etoolbox}
\makeatletter
\providecommand{\subtitle}[1]{% add subtitle to \maketitle
  \apptocmd{\@title}{\par {\large #1 \par}}{}{}
}
\makeatother
\subtitle{114 様式04}
\author{}
\date{2025-09-08}
\begin{document}
\maketitle


\% 手動でタイトルを作成

\section{様式04:
患者安全確認表}\label{ux69d8ux5f0f04-ux60a3ux8005ux5b89ux5168ux78baux8a8dux8868}

\textbf{【発災日時】}\underlinespace{2cm}年\underlinespace{1cm}月\underlinespace{1cm}日\underlinespace{1cm}時\underlinespace{1cm}分

\textbf{【確認日時】}\underlinespace{2cm}年\underlinespace{1cm}月\underlinespace{1cm}日\underlinespace{1cm}時\underlinespace{1cm}分

\textbf{【施設名】} \underlinespace{10cm}

\textbf{【確認者】} \underlinespace{10cm}

\% 発災時施設内患者状況

\textbf{\large 発災時施設内患者状況}

\checkbox 透析中患者数:\underlinespace{2cm}名

\checkbox 待機中患者数:\underlinespace{2cm}名

\checkbox その他患者数:\underlinespace{2cm}名

\% 患者安全確認状況表

\textbf{\large 患者安全確認状況}

\begin{longtable*}{|p{1.8cm}|p{2.2cm}|p{2cm}|p{2cm}|p{2.2cm}|p{1.8cm}|p{2cm}|}
\hline
**患者ID** & **氏名** & **状態** & **負傷の有無** & **処置内容** & **搬送要否** & **備考** \\
\hline
\endfirsthead

\hline
**患者ID** & **氏名** & **状態** & **負傷の有無** & **処置内容** & **搬送要否** & **備考** \\
\hline
\endhead

 &  & 透析中/待機/その他 & 無/軽傷/重傷 &  & 不要/要/済 &  \\
\hline
 &  & 透析中/待機/その他 & 無/軽傷/重傷 &  & 不要/要/済 &  \\
\hline
 &  & 透析中/待機/その他 & 無/軽傷/重傷 &  & 不要/要/済 &  \\
\hline
 &  & 透析中/待機/その他 & 無/軽傷/重傷 &  & 不要/要/済 &  \\
\hline
 &  & 透析中/待機/その他 & 無/軽傷/重傷 &  & 不要/要/済 &  \\
\hline
 &  & 透析中/待機/その他 & 無/軽傷/重傷 &  & 不要/要/済 &  \\
\hline
 &  & 透析中/待機/その他 & 無/軽傷/重傷 &  & 不要/要/済 &  \\
\hline
 &  & 透析中/待機/その他 & 無/軽傷/重傷 &  & 不要/要/済 &  \\
\hline
 &  & 透析中/待機/その他 & 無/軽傷/重傷 &  & 不要/要/済 &  \\
\hline
 &  & 透析中/待機/その他 & 無/軽傷/重傷 &  & 不要/要/済 &  \\
\hline
 &  & 透析中/待機/その他 & 無/軽傷/重傷 &  & 不要/要/済 &  \\
\hline
 &  & 透析中/待機/その他 & 無/軽傷/重傷 &  & 不要/要/済 &  \\
\hline
 &  & 透析中/待機/その他 & 無/軽傷/重傷 &  & 不要/要/済 &  \\
\hline
 &  & 透析中/待機/その他 & 無/軽傷/重傷 &  & 不要/要/済 &  \\
\hline
 &  & 透析中/待機/その他 & 無/軽傷/重傷 &  & 不要/要/済 &  \\
\hline
\end{longtable*}

\% 透析治療継続状況

\textbf{\large 透析治療継続状況}

\checkbox 透析継続可能患者:\underlinespace{2cm}名

\checkbox 透析中止患者:\underlinespace{2cm}名(理由:\underlinespace{6cm})

\checkbox 緊急搬送患者:\underlinespace{2cm}名

\% 心理的ケア実施状況

\textbf{\large 心理的ケア実施状況}

\checkbox パニック状態患者:\underlinespace{2cm}名(対応:\underlinespace{6cm})

\checkbox 不安状態患者:\underlinespace{2cm}名(対応:\underlinespace{6cm})

\checkbox 特別配慮必要患者:\underlinespace{2cm}名(内容:\underlinespace{6cm})

\% 集計

\textbf{\large 集計}

\textbf{総患者数:}\underlinespace{2cm}名

\textbf{安全確認済み:}\underlinespace{2cm}名(うち無事:\underlinespace{1.5cm}名、軽傷:\underlinespace{1.5cm}名、重傷:\underlinespace{1.5cm}名)

\textbf{搬送実施:}\underlinespace{2cm}名

\textbf{治療継続:}\underlinespace{2cm}名

\% 特記事項

\textbf{\large 特記事項}

\underlinespace{13cm}

\underlinespace{13cm}

\underlinespace{13cm}

\underlinespace{13cm}

\underlinespace{13cm}




\end{document}
