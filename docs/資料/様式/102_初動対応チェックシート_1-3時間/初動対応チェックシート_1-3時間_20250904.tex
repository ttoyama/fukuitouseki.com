% 初動対応チェックシート(発災後1~3時間)
% LaTeX template for disaster dialysis manual forms

\documentclass[a4paper,12pt]{jarticle}
\usepackage[top=30mm, bottom=30mm, left=20mm, right=20mm, footskip=18mm, headsep=12mm]{geometry}
\usepackage{setspace}
\setstretch{1.3}
\usepackage{array}
\usepackage{longtable}
\usepackage{amssymb}
\usepackage{multirow}
\usepackage{booktabs}

\newcommand{\checkbox}{$\square$\ }
\newcommand{\checkedbox}{$\blacksquare$\ }
\newcommand{\underlinespace}[1]{\underline{\hspace{#1}}}
\newcommand{\circlecheck}{$\bigcirc$\ }

% シンプルなページ番号設定
\pagestyle{plain}
\makeatletter
\def\@oddhead{\hfill\small 102 初動対応チェックシート(1-3時間) 2025.09.04版}
\def\@evenhead{\hfill\small 102 初動対応チェックシート(1-3時間) 2025.09.04版}
\def\@oddfoot{\hfil\thepage\hfil}
\def\@evenfoot{\hfil\thepage\hfil}
\makeatother

\begin{document}

% 手動でタイトルを作成
\begin{center}
{\Large\textbf{初動対応チェックシート(発災後1~3時間)}}
\end{center}
\vspace{5mm}

\noindent
\textbf{施設名:} \underlinespace{8cm}

\vspace{3mm}

\noindent
\textbf{記録者:} \underlinespace{4cm}

\vspace{3mm}

\noindent
\textbf{記録日時:} \underlinespace{2cm}年\underlinespace{1cm}月\underlinespace{1cm}日 \quad \circlecheck 午前 \quad \circlecheck 午後 \quad \underlinespace{1cm}時\underlinespace{1cm}分

\vspace{5mm}

\section*{1. 内部体制確立}

\checkbox \textbf{施設内災害対策本部を立ち上げる}

\quad 立ち上げ完了時刻:\underlinespace{1cm}時\underlinespace{1cm}分 \quad 本部長:\underlinespace{4cm}

\vspace{3mm}

\checkbox \textbf{状況共有・指示系統を確立する}

\quad 体制確立時刻:\underlinespace{1cm}時\underlinespace{1cm}分 \quad 参加スタッフ数:\underlinespace{2cm}人

\vspace{3mm}

\checkbox \textbf{通信手段を確保する}

\quad \checkbox 固定電話 \quad \checkbox 携帯電話 \quad \checkbox 衛星電話 \quad \checkbox 防災無線

\quad 利用可能通信手段:\underlinespace{10cm}

\vspace{5mm}

\section*{2. 詳細被災状況報告(発災後1時間)}

\subsection*{2-1. 福井県透析施設ネットワーク本部への報告}

\checkbox \textbf{透析設備の損傷・稼働状況を報告する}

\quad 報告完了時刻:\underlinespace{1cm}時\underlinespace{1cm}分 \quad 報告者:\underlinespace{4cm}

\quad 稼働可能台数:\underlinespace{2cm}台/総\underlinespace{2cm}台

\quad 損傷詳細:\underlinespace{10cm}

\vspace{4mm}

\checkbox \textbf{ライフラインの使用可否を報告する}

\quad \textbf{電気:} \circlecheck 使用可能 \circlecheck 一部制限 \circlecheck 使用不可

\quad \textbf{水道:} \circlecheck 使用可能 \circlecheck 一部制限 \circlecheck 使用不可

\quad \textbf{ガス:} \circlecheck 使用可能 \circlecheck 一部制限 \circlecheck 使用不可

\quad \textbf{通信:} \circlecheck 使用可能 \circlecheck 一部制限 \circlecheck 使用不可

\vspace{4mm}

\checkbox \textbf{施設へのアクセス道路状況を報告する}

\quad 主要アクセス路:\circlecheck 通行可能 \circlecheck 通行困難 \circlecheck 通行不可

\quad 詳細状況:\underlinespace{10cm}

\vspace{4mm}

\checkbox \textbf{医薬品・医療材料の被害状況を報告する}

\quad \circlecheck 問題なし \quad \circlecheck 一部破損 \quad \circlecheck 大幅破損

\quad 破損詳細:\underlinespace{10cm}

\vspace{5mm}

\newpage

\section*{3. 運営判断・支援要請準備}

\checkbox \textbf{透析継続の可否を判断する}

\quad \circlecheck 継続可能 \quad \circlecheck 一部制限あり \quad \circlecheck 継続不可

\quad 判断根拠:\underlinespace{10cm}

\vspace{4mm}

\checkbox \textbf{支援が必要な患者数を算出し緊急度を分類する}

\quad 高緊急度(赤):\underlinespace{2cm}人 \quad 中緊急度(黄):\underlinespace{2cm}人 \quad 低緊急度(緑):\underlinespace{2cm}人

\vspace{4mm}

\checkbox \textbf{必要支援内容を整理する}

\quad \checkbox 人員支援 \quad \checkbox 物資支援 \quad \checkbox 患者搬送

\quad 詳細要請内容:\underlinespace{10cm}

\vspace{4mm}

\checkbox \textbf{受け入れ可能患者数を算定する}

\quad 緊急受入:\underlinespace{2cm}人 \quad 当日受入:\underlinespace{2cm}人 \quad 翌日受入:\underlinespace{2cm}人

\vspace{5mm}

\section*{4. 運営可否・支援要請報告(発災後3時間)}

\subsection*{4-1. 福井県透析施設ネットワーク本部への報告}

\checkbox \textbf{透析継続の可否と理由を報告する}

\quad 報告完了時刻:\underlinespace{1cm}時\underlinespace{1cm}分 \quad 報告者:\underlinespace{4cm}

\quad 継続判断:\circlecheck 継続可能 \circlecheck 一部制限 \circlecheck 継続不可

\quad 理由:\underlinespace{10cm}

\vspace{4mm}

\checkbox \textbf{支援が必要な患者数と緊急度分類を報告する}

\quad 支援必要患者総数:\underlinespace{2cm}人

\quad 内訳:高緊急度\underlinespace{2cm}人、中緊急度\underlinespace{2cm}人、低緊急度\underlinespace{2cm}人

\vspace{4mm}

\checkbox \textbf{必要な支援内容を報告する}

\quad \checkbox 人員派遣要請 \quad \checkbox 物資支援要請 \quad \checkbox 患者搬送要請

\quad 詳細要請内容:\underlinespace{10cm}

\vspace{4mm}

\checkbox \textbf{受け入れ可能患者数を報告する}

\quad 受入可能総数:\underlinespace{2cm}人

\quad 内訳:緊急\underlinespace{2cm}人、当日\underlinespace{2cm}人、翌日\underlinespace{2cm}人

\vspace{5mm}

\newpage

\section*{5. 初動対応記録}

\checkbox \textbf{初動対応記録を作成・更新する}

\subsection*{5-1. 災害発生からの状況記録}

\textbf{発災時刻:} \underlinespace{1cm}時\underlinespace{1cm}分 \quad \textbf{災害種別:} \underlinespace{6cm}

\vspace{3mm}

\textbf{主な被害状況:}

\underlinespace{15cm}

\underlinespace{15cm}

\vspace{3mm}

\subsection*{5-2. 実施した対応・判断内容記録}

\textbf{時刻:\underlinespace{1cm}時\underlinespace{1cm}分} \quad \textbf{対応内容:} \underlinespace{10cm}

\textbf{時刻:\underlinespace{1cm}時\underlinespace{1cm}分} \quad \textbf{対応内容:} \underlinespace{10cm}

\textbf{時刻:\underlinespace{1cm}時\underlinespace{1cm}分} \quad \textbf{対応内容:} \underlinespace{10cm}

\textbf{時刻:\underlinespace{1cm}時\underlinespace{1cm}分} \quad \textbf{対応内容:} \underlinespace{10cm}

\vspace{3mm}

\subsection*{5-3. スタッフ行動記録}

\textbf{スタッフ名:} \underlinespace{4cm} \quad \textbf{主な対応:} \underlinespace{8cm}

\textbf{スタッフ名:} \underlinespace{4cm} \quad \textbf{主な対応:} \underlinespace{8cm}

\textbf{スタッフ名:} \underlinespace{4cm} \quad \textbf{主な対応:} \underlinespace{8cm}

\vspace{3mm}

\subsection*{5-4. 患者状況変化記録}

\textbf{患者名:} \underlinespace{3cm} \quad \textbf{状況変化:} \underlinespace{9cm}

\textbf{患者名:} \underlinespace{3cm} \quad \textbf{状況変化:} \underlinespace{9cm}

\textbf{患者名:} \underlinespace{3cm} \quad \textbf{状況変化:} \underlinespace{9cm}

\vspace{3mm}

\subsection*{5-5. 外部連絡記録}

\textbf{時刻:\underlinespace{1cm}時\underlinespace{1cm}分} \quad \textbf{連絡先:} \underlinespace{5cm} \quad \textbf{内容:} \underlinespace{6cm}

\textbf{時刻:\underlinespace{1cm}時\underlinespace{1cm}分} \quad \textbf{連絡先:} \underlinespace{5cm} \quad \textbf{内容:} \underlinespace{6cm}

\textbf{時刻:\underlinespace{1cm}時\underlinespace{1cm}分} \quad \textbf{連絡先:} \underlinespace{5cm} \quad \textbf{内容:} \underlinespace{6cm}

\vspace{5mm}

\noindent
\textbf{総合確認:} \\
\checkbox 内部体制確立完了 \\
\checkbox 詳細被災状況報告(1時間)完了 \\
\checkbox 運営判断・支援要請準備完了 \\
\checkbox 運営可否・支援要請報告(3時間)完了 \\
\checkbox 初動対応記録作成・更新完了

\vspace{5mm}

\noindent
最終確認者:\underlinespace{4cm} \\
\vspace{3mm}
確認日時:\underlinespace{2cm}年\underlinespace{1cm}月\underlinespace{1cm}日 \quad \circlecheck 午前 \quad \circlecheck 午後 \quad \underlinespace{1cm}時\underlinespace{1cm}分

\end{document}